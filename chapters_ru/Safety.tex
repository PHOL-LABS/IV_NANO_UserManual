\chapter{Техника безопасности}\label{ch:safety}

Оборудование IV-Indicators Nano чувствительно к коротким замыканиям и неверному подключению датчиков. Перед подачей питания на программатор обязательно ознакомьтесь с запретами в \autoref{ch:precautions} и следуйте рекомендациям ниже при установке.

\section{Работа с программатором}
\begin{itemize}
    \item Относитесь к программатору как к части индикатора. Не подавайте питание, пока разъём-пого не установлен и не опирается на непроводящую поверхность.
    \item Не подключайте и не отсоединяйте программатор во время передачи данных. Сначала извлеките USB-кабель и удерживайте устройство неподвижно, чтобы не погнуть контакты.
    \item Избегайте проводящих рабочих поверхностей: корпусов ноутбуков, голого металла или кузова автомобиля. Даже кратковременное касание может замкнуть контакты и окончательно повредить электронику.
\end{itemize}

\section{Безопасность при работе с датчиками}
\begin{itemize}
    \item Используйте рекомендуемые разъёмы для каждого пресета и фиксируйте жгуты от натяжения. Для барометра требуется разъём VAG \texttt{8K0973703}, а пресет наддува следует подключать через герметичный трёхконтактный разъём.
    \item При установке индикатора лямбда используйте поддерживаемый широкополосный контроллер на доработанных моторах. Контроллеры SLC Free, DIY-EFI TinyWB или Sigma Lambda Controller Free~2 преобразуют сигнал Bosch LSU~4.9 в безопасное аналоговое напряжение для индикатора.
    \item Перед поездкой убедитесь, что пределы калибровки соответствуют установленному датчику. Неверные границы приводят к искажённым показаниям и ошибочной диагностике.
\end{itemize}

\section{Гарантийное примечание}
PHOL-LABS Kft не предоставляет гарантий на устройства, которые включались на проводящих поверхностях, подключались с перепутанной полярностью или эксплуатировались в нарушение перечисленных правил. Замена возможна, но при выявлении неправильной эксплуатации она предоставляется на платной основе.
