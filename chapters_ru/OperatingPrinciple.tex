\chapter{Принцип работы}\label{ch:operating-principle}

Каждый индикатор IV-Indicators Nano сочетает сегментный дисплей, RGB-подсветку и микроконтроллер, в котором хранятся пресеты измерений. На задней стороне расположен четырёхконтактный разъём-пого, куда подключается комплектный программатор, чтобы индикатор мог получать обновления прошивки и профили конфигурации с Android-смартфона или компьютера через браузер.

\section{Пресеты измерений}
Каждый индикатор поставляется с пресетом под свою задачу:
\begin{itemize}
    \item \textbf{Voltmeter} --- контролирует бортовое напряжение напрямую от проводки автомобиля.
    \item \textbf{Barometer} --- рассчитан на датчик давления VAG 03C906051A с питанием 5~В.
    \item \textbf{Thermometer} --- считывает NTC-терморезисторы, эквивалентные датчику Ossca\,01176, приведённому в \autoref{app:temperature-table}.
    \item \textbf{Lambda} --- воспринимает сигналы узкополосного кислородного датчика и принимает линейный выход 0--5~В от широкополосных контроллеров.
    \item \textbf{Boost} --- работает с датчиком Dacia 223657266R\\ (также известным как 161B0004/8200225971).
\end{itemize}

\section{Процесс конфигурирования}
Данные конфигурации передаются по USB через приложения IV-Conf. Веб-пакет и Android-бета предоставляют одинаковые настройки цветов, градиентов и режимов отображения, чтобы одно и то же устройство могло работать в режимах точек или столбиков. Можно задать до четырёх цветовых сегментов, отрегулировать яркость подсветки и настроить минимальные/максимальные диапазоны измерений под используемые датчики. Откалиброванные значения сохраняются даже после отключения индикатора, поэтому можно свободно экспериментировать, не теряя понравившиеся настройки.

\section{Интеграция датчиков}
Пресет барометра рассчитан на штатный разъём VAG \texttt{8K0973703}. Индикатор наддува поставляется без отдельного жгута; рекомендуется использовать герметичный трёхконтактный автомобильный разъём, например Bosch \texttt{1928403966}. Для контроля лямбда-зонда PHOL-LABS рекомендует использовать внешний широкополосный контроллер (SLC Free, DIY-EFI TinyWB, Sigma Lambda Controller Free~2 или аналогичный), который преобразует сигнал датчика Bosch LSU~4.9 в стабильное аналоговое напряжение. Передавайте линейный выход контроллера на вход индикатора, соблюдая инструкцию контроллера по питанию нагревателя и калибровке.
