\chapter{Типовые сценарии настройки}\label{ch:setup-cases}

Перед любыми изменениями установите стабильное соединение по MODBUS (без красных предупреждений). Скриншоты взяты из веб-пакета IV-Conf, но в Android-бете доступны аналогичные органы управления.

\section{Настройка цветов}
\begin{enumerate}
    \item Нажмите \texttt{Get data}, чтобы считать текущий пресет.
    \item Измените параметры hue, чтобы задать нужный цвет индикации. Текст предпросмотра соответствует выходу индикатора.
    \item Отрегулируйте бегунок \texttt{Bl} для настройки яркости подсветки.
    \item Нажмите \texttt{Send data} и дождитесь завершения передачи.
\end{enumerate}

\section{Калибровка диапазона}
\begin{enumerate}
    \item Считайте конфигурацию через \texttt{Get data}.
    \item Установите \texttt{Range min} и \texttt{Range max} так, чтобы индикатор отражал реальный диапазон датчика (вольтметр, барометр, лямбда или наддув).
    \item Отправьте данные и сравните текущее показание с эталонным прибором.
\end{enumerate}

\section{Настройка термометра}
\begin{enumerate}
    \item Измерьте сопротивление датчика мультиметром и сравните его со значениями в \autoref{app:temperature-table}. При выходе за допуски замените датчик.
    \item Нажмите \texttt{Get data}, чтобы загрузить текущие параметры.
    \item Введите соответствующий коэффициент бета и номинальное сопротивление для пользовательского NTC-датчика.
    \item Запишите изменения через \texttt{Send data}.
\end{enumerate}

\section{Переключение режимов столбиков и точек}
\begin{enumerate}
    \item Получите конфигурацию с помощью \texttt{Get data}.
    \item Установите \texttt{Bar mode} в 1 для столбикового отображения или сбросьте для индикации точками.
    \item Нажмите \texttt{Send data}, чтобы применить изменение.
\end{enumerate}

\section{Ключевые элементы интерфейса}
\begin{figure}[htbp]
    \centering
    \begin{subfigure}{0.45\textwidth}
        \centering
        \includegraphics[width=\textwidth]{image29.png}
        \caption{Редактор градиента и оттенка}
    \end{subfigure}\hfill
    \begin{subfigure}{0.45\textwidth}
        \centering
        \includegraphics[width=\textwidth]{image26.png}
        \caption{Поля калибровки диапазона}
    \end{subfigure}
    \par\medskip
    \begin{subfigure}{0.45\textwidth}
        \centering
        \includegraphics[width=\textwidth]{image24.png}
        \caption{Настройка пользовательского NTC}
    \end{subfigure}\hfill
    \begin{subfigure}{0.45\textwidth}
        \centering
        \includegraphics[width=\textwidth]{image4.png}
        \caption{Переключатель столбикового режима}
    \end{subfigure}
    \caption{Основные элементы IV-Conf, на которые ссылаются типовые сценарии.}
\end{figure}
