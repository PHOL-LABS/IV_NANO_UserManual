\chapter{Приложение}\label{ch:appendix}

\section{Перепрошивка через веб-пакет IV-Conf}\label{app:reflashing}
Для обновления прошивки используйте веб-пакет. Выполните шаги ниже и не отключайте питание после начала передачи.
\begin{enumerate}
    \item Откройте \url{https://phol-labs.com/iv} и подключитесь к программатору по инструкции в \autoref{ch:web-config}.
    \item Убедитесь, что скорость установлена 115200 бод, и нажмите \texttt{Connect MODBUS}. Дождитесь зелёного статуса.
    \item Нажмите \texttt{Stop Send Data}, чтобы остановить автоматическую загрузку пресетов.
    \item Кликните на значок прошивки, выберите файл обновления, полученный с индикатором, и подтвердите действие.
    \item Не трогайте индикатор, пока полоса прогресса не завершится и устройство не перезапустится.
\end{enumerate}

\begin{figure}[htbp]
    \centering
    \begin{subfigure}{0.3\textwidth}
        \centering
        \includegraphics[width=\textwidth]{image15.png}
        \caption{Открытие веб-пакета}
    \end{subfigure}\hfill
    \begin{subfigure}{0.3\textwidth}
        \centering
        \includegraphics[width=\textwidth]{image11.png}
        \caption{Выбор последовательного порта}
    \end{subfigure}\hfill
    \begin{subfigure}{0.3\textwidth}
        \centering
        \includegraphics[width=\textwidth]{image1.png}
        \caption{Подтверждение доступа}
    \end{subfigure}
    \par\medskip
    \begin{subfigure}{0.3\textwidth}
        \centering
        \includegraphics[width=\textwidth]{image8.png}
        \caption{Установка скорости 115200}
    \end{subfigure}\hfill
    \begin{subfigure}{0.3\textwidth}
        \centering
        \includegraphics[width=\textwidth]{image25.png}
        \caption{Остановка активных передач}
    \end{subfigure}\hfill
    \begin{subfigure}{0.3\textwidth}
        \centering
        \includegraphics[width=\textwidth]{image14.png}
        \caption{Соединение по MODBUS}
    \end{subfigure}
    \par\medskip
    \begin{subfigure}{0.3\textwidth}
        \centering
        \includegraphics[width=\textwidth]{image6.png}
        \caption{Открытие диалога прошивки}
    \end{subfigure}\hfill
    \begin{subfigure}{0.3\textwidth}
        \centering
        \includegraphics[width=\textwidth]{image46.png}
        \caption{Выбор пакета обновления}
    \end{subfigure}\hfill
    \begin{subfigure}{0.3\textwidth}
        \centering
        \includegraphics[width=\textwidth]{image7.png}
        \caption{Подтверждение загрузки}
    \end{subfigure}
    \par\medskip
    \begin{subfigure}{0.3\textwidth}
        \centering
        \includegraphics[width=\textwidth]{image12.png}
        \caption{Ожидание завершения}
    \end{subfigure}
    \caption{Последовательность экранов веб-пакета IV-Conf при перепрошивке.}
\end{figure}

\section{Справочные данные по датчику температуры}\label{app:temperature-table}
Ниже приведена таблица номинальных сопротивлений терморезистора Ossca \texttt{01176}, используемого в пресете термометра.

\begin{table}[htbp]
    \centering
    \caption{Эталонные сопротивления NTC}
    \label{tab:ntc-reference}
    \begin{tabular}{@{} rr @{}}
        \toprule
        Сопротивление (\si{\ohm}) & Температура (\si{\celsius}) \\
        \midrule
        270.0 & 58 \\
        220.0 & 63 \\
        199.8 & 66 \\
        111.0 & 83 \\
        73.8 & 98 \\
        55.0 & 108 \\
        48.8 & 113 \\
        44.0 & 117 \\
        37.2 & 124 \\
        32.1 & 130 \\
        28.2 & 136 \\
        25.1 & 141 \\
        22.7 & 146 \\
        20.4 & 151 \\
        19.0 & 155 \\
        18.8 & 155 \\
        15.9 & 160 \\
        \bottomrule
    \end{tabular}
\end{table}

В будущих версиях приложения будут добавлены дополнительные данные по датчикам и схемам проводки.
