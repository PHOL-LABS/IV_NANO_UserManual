\chapter{Технические характеристики}\label{ch:technical-specifications}

\section{Поддерживаемые датчики}

\begin{table}[htbp]
    \centering
    \caption{Заводские пресеты и совместимые датчики}
    \label{tab:sensor-characteristics}

    \begin{tabularx}{\textwidth}{@{} l l l X @{}}
        \toprule
        Индикатор & Датчик & Питание & Примечания \\
        \midrule
        Voltmeter & Бортовая сеть & 12~В & Прямое измерение напряжения зарядки. \\
        Barometer & VAG \texttt{03C906051A} & 5~В &
            Диапазон 10~бар, номинальный выход 0.4--0.5~В, резьба M10\,\texttimes\,1.0. \\
        Thermometer & NTC Ossca \texttt{01176} & Пассивный &
            Опорные значения приведены в \autoref{app:temperature-table}. \\
        Lambda & Узкополосный O$_2$ датчик & Пассивный &
            Размах 0.1--0.9~В, принимает 0--5~В от широкополосных контроллеров. \\
        Boost & Dacia \texttt{223657266R} & 5~В &
            Диапазон 20--250~кПа, номинальный выход 1.8~В, одно крепёжное отверстие. \\
        \bottomrule
\end{tabularx}
\end{table}

\section{Таблицы совместимости датчиков}
В таблицах ниже приведены совместимые датчики и опорные параметры калибровки для заводских пресетов.

\subsection{Датчики температуры масла/охлаждающей жидкости}
\begin{table}[htbp]
    \centering
    \caption{Совместимость датчиков температуры масла/охлаждающей жидкости}
    \label{tab:oil-coolant-sensors}
    \begin{tabularx}{\textwidth}{@{} l l X @{}}
        \toprule
        Совместимый датчик & Сигнал/питание & Параметры калибровки \\
        \midrule
        Терморезистор Ossca \texttt{01176} NTC (пресет термометра) & Пассивный NTC &
            Проверяйте по таблице сопротивлений Ossca \texttt{01176} (например, 270{,}0~$\Omega$ при 58~$^\circ$C до 15{,}9~$\Omega$ при 160~$^\circ$C). Номинальная калибровка: $R_{25} = 1$~k$\Omega$, $\beta = 3950$. Для пользовательских NTC обновите коэффициент бета и номинальное сопротивление в IV-Conf. \\
        RIDEX \texttt{829S0003} (совместим с Opel/BMW/Volvo) & 2-контактный NTC, резьба M12\,\texttimes\,1.5 &
            Номинальное сопротивление 2080~$\Omega$ при 25~$^\circ$C и 294~$\Omega$ при 80~$^\circ$C ($\beta \approx 3740$~K). Рабочий диапазон 25--80~$^\circ$C. Корпус $\varnothing$~7{,}4~мм, шестигранник 19~мм, поставляется с уплотнительной прокладкой. \\
        \bottomrule
    \end{tabularx}
\end{table}

\subsection{Датчики температуры воздуха}
\begin{table}[htbp]
    \centering
    \caption{Совместимость датчиков температуры воздуха}
    \label{tab:air-temperature-sensors}
    \begin{tabularx}{\textwidth}{@{} l l X @{}}
        \toprule
        Совместимый датчик & Сигнал/питание & Параметры калибровки \\
        \midrule
        Датчик наружной температуры MFA (VAG \texttt{171 919 379 A}, Golf Mk2/Jetta II/Passat B2/B3) & 2-проводной NTC, плавающий &
            Диапазон отображения примерно от $-40^\circ$C до $+96^\circ$C. Эмпирическая калибровка: $R_{25} \approx 510~\Omega$, $\beta \approx 3400$--3500~K (подгонка между $+4^\circ$C и $+50^\circ$C). Диагностическая точка: 200~$\Omega \approx +50^\circ$C. \\
        Терморезистор NTC, эквивалентный Ossca \texttt{01176} (пресет термометра) & Пассивный NTC &
            Используйте таблицу сопротивлений Ossca \texttt{01176} в качестве опорной и настройте бета/номинальное сопротивление для пользовательских NTC через IV-Conf. \\
        \bottomrule
    \end{tabularx}
\end{table}

\subsection{Датчики давления масла}
\begin{table}[htbp]
    \centering
    \caption{Совместимость датчиков давления масла}
    \label{tab:oil-pressure-sensors}
    \begin{tabularx}{\textwidth}{@{} l l X @{}}
        \toprule
        Совместимый датчик & Сигнал/питание & Параметры калибровки \\
        \midrule
        Датчик давления VAG \texttt{03C906051A} (пресет барометра) & Питание 5~В, аналоговый выход &
            Абсолютный диапазон 10~бар, номинальный выход 0.4--0.5~В в покое. Настройте измерительный диапазон через пресет барометра и при необходимости скорректируйте минимум/максимум в IV-Conf. \\
        \bottomrule
    \end{tabularx}
\end{table}

\subsection{Датчики уровня топлива}
\begin{table}[htbp]
    \centering
    \caption{Совместимость датчиков уровня топлива}
    \label{tab:fuel-level-sensors}
    \begin{tabularx}{\textwidth}{@{} l l X @{}}
        \toprule
        Совместимый датчик & Сигнал/питание & Параметры калибровки \\
        \midrule
        Пользовательский датчик (модель не указана) & Зависит от датчика &
            Задайте минимальный и максимальный диапазоны измерения в IV-Conf под установленный датчик. \\
        \bottomrule
    \end{tabularx}
\end{table}

\subsection{Датчики давления наддува}
\begin{table}[htbp]
    \centering
    \caption{Совместимость датчиков давления наддува}
    \label{tab:boost-pressure-sensors}
    \begin{tabularx}{\textwidth}{@{} l l X @{}}
        \toprule
        Совместимый датчик & Сигнал/питание & Параметры калибровки \\
        \midrule
        Dacia \texttt{223657266R} (также \texttt{161B0004}/\texttt{8200225971}) & Питание 5~В, аналоговый выход &
            Окно работы 20--250~кПа, номинальный выход $\sim$1{,}8~В при наддуве. Используйте пресет наддува и при необходимости скорректируйте range min/max. \\
        \bottomrule
    \end{tabularx}
\end{table}

\subsection{Широкополосная лямбда}
\begin{table}[htbp]
    \centering
    \caption{Совместимость контроллеров широкополосной лямбды}
    \label{tab:wideband-lambda}
    \begin{tabularx}{\textwidth}{@{} l l X @{}}
        \toprule
        Совместимый датчик & Сигнал/питание & Параметры калибровки \\
        \midrule
        Широкополосный контроллер (SLC Free, DIY-EFI TinyWB, Sigma Lambda Controller Free~2) с датчиком Bosch LSU~4.9 & Линейный выход 0--5~В на индикатор &
            Широкополосные контроллеры преобразуют LSU~4.9 в линейный сигнал 0--5~В, принимаемый индикатором; узкополосные датчики дают диапазон 0.1--0.9~В. \\
        \bottomrule
    \end{tabularx}
\end{table}

\section{Пресет барометра}
Индикатор барометра рассчитан на датчик давления \texttt{03C906051A} группы Volkswagen.
\begin{itemize}
    \item Питание: стабилизированные 5~В.
    \item Диапазон измерений: 10~бар абсолютного давления.
    \item Номинальный аналоговый выход: 0.4--0.5~В в покое.
    \item Механическая резьба: M10\,\texttimes\,1.0.
    \item Разъём: VAG \texttt{8K0973703}.
\end{itemize}

\section{Пресет термометра}
Пресеты термометра откалиброваны под терморезистор Ossca\,01176, применяемый в системах охлаждения Golf~II. При проверке альтернативных датчиков сверяйте сопротивление с таблицей в \autoref{app:temperature-table}. Для пользовательских NTC укажите коэффициент бета и номинальное сопротивление через утилиты IV-Conf.

\section{Пресет наддува}
Индикатор наддува рассчитан на датчик Dacia \texttt{223657266R} (\texttt{161B0004}/\texttt{8200225971}).
\begin{itemize}
    \item Трёхконтактный разъём (используйте герметичный Bosch \texttt{1928403966} или эквивалент).\!
    \item Напряжение питания: +5~В.
    \item Рабочий диапазон: 20--250~кПа.
    \item Аналоговый выход: примерно +1.8~В при номинальном давлении.
\end{itemize}

\section{Интеграция лямбда-зонда}
IV-Indicators Lambda может считывать узкополосные датчики напрямую, но предназначен для работы совместно с широкополосным контроллером на доработанных моторах. Рекомендуемые контроллеры: SLC Free, DIY-EFI TinyWB и Sigma Lambda Controller Free~2. Каждый из них преобразует сигнал датчика Bosch LSU~4.9 в линейное напряжение 0--5~В, которое отображает индикатор.

\section{Справочные изображения датчиков}
\begin{figure}[htbp]
    \centering
    \begin{subfigure}{0.3\textwidth}
        \centering
        \includegraphics[width=\textwidth]{image41.jpg}
        \caption{Датчик давления VAG 03C906051A}
    \end{subfigure}\hfill
    \begin{subfigure}{0.3\textwidth}
        \centering
        \includegraphics[width=\textwidth]{image9.png}
        \caption{Распиновка жгута барометра}
    \end{subfigure}\hfill
    \begin{subfigure}{0.3\textwidth}
        \centering
        \includegraphics[width=\textwidth]{image42.jpg}
        \caption{Пример подключения датчика наддува}
    \end{subfigure}
    \caption{Рекомендованное оборудование для пресетов барометра и наддува.}
\end{figure}
