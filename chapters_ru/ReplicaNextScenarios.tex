\chapter{Конфигурирование через веб-пакет IV-Conf}\label{ch:web-config}

Веб-пакет IV-Conf — основной инструмент настройки оборудования IV-Indicators Nano. Откройте \url{https://phol-labs.com/iv} в браузере на базе Chromium с поддержкой Web Serial и подключите комплектный программатор по USB.

\section{Подключение к индикатору}
\begin{enumerate}
    \item Нажмите на значок последовательного порта и выберите запись, соответствующую программатору.
    \item Откройте панель \texttt{Settings} и установите скорость 115200 бод.
    \item Нажмите \texttt{Connect MODBUS}. В статусной строке должно появиться подтверждение соединения без красных ошибок.
\end{enumerate}

\section{Каталог пресетов}
Для каждого типа индикатора доступно несколько пресетов. Например, у вольтметра есть варианты с многосегментной графикой или жёлтой подсветкой. Перед загрузкой выберите пресет, соответствующий установленному прибору.

\begin{table}[htbp]
    \centering
    \caption{Примеры пресетов вольтметра}
    \label{tab:voltmeter-presets}
    \begin{tabular}{@{} l l @{}}
        \toprule
        Название пресета & Описание \\
        \midrule
        IV-Volts-Bar-Multi & Вольтметр с многосегментной индикацией.\\
        IV-Volts-Bar-Multi4 & Вариант с четырьмя уровнями многосегментного отображения.\\
        IV-Volts-BarYellow & Столбиковая индикация с жёлтой цветовой схемой.\\
        IV-Volts-Bar & Стандартное отображение столбиками.\\
        IV-Volts & Минимальный пресет без дополнительной графики.\\
        \bottomrule
    \end{tabular}
\end{table}

\section{Загрузка и чтение данных}
\begin{itemize}
    \item Кнопка \texttt{Send data} отправляет текущий пресет или ручные изменения в индикатор. Передача завершается автоматически за 10--15~секунд.
    \item Нажмите \texttt{Get data}, чтобы считать текущую конфигурацию прибора перед изменением диапазонов или цветов.
    \item Кнопка \texttt{Stop Send Data} прерывает передачу, если выбран неверный пресет или нужно изменить параметры.
\end{itemize}

Ручная настройка позволяет подбирать начальный и конечный цвет градиента, яркость подсветки и числовые диапазоны датчиков под нестандартное оборудование. Убедитесь, что каждое изменение соответствует установленному индикатору, чтобы избежать неверных показаний.

\section{Обзор интерфейса}
\begin{figure}[htbp]
    \centering
    \begin{subfigure}{0.3\textwidth}
        \centering
        \includegraphics[width=\textwidth]{image15.png}
        \caption{Запуск веб-пакета}
    \end{subfigure}\hfill
    \begin{subfigure}{0.3\textwidth}
        \centering
        \includegraphics[width=\textwidth]{image11.png}
        \caption{Выбор последовательного порта}
    \end{subfigure}\hfill
    \begin{subfigure}{0.3\textwidth}
        \centering
        \includegraphics[width=\textwidth]{image1.png}
        \caption{Предоставление доступа к порту}
    \end{subfigure}
    \par\medskip
    \begin{subfigure}{0.3\textwidth}
        \centering
        \includegraphics[width=\textwidth]{image8.png}
        \caption{Настройка скорости 115200}
    \end{subfigure}\hfill
    \begin{subfigure}{0.3\textwidth}
        \centering
        \includegraphics[width=\textwidth]{image22.png}
        \caption{Соединение по MODBUS}
    \end{subfigure}\hfill
    \begin{subfigure}{0.3\textwidth}
        \centering
        \includegraphics[width=\textwidth]{image5.png}
        \caption{Выбор пресета}
    \end{subfigure}
    \par\medskip
    \begin{subfigure}{0.3\textwidth}
        \centering
        \includegraphics[width=\textwidth]{image3.png}
        \caption{Ручные настройки}
    \end{subfigure}\hfill
    \begin{subfigure}{0.3\textwidth}
        \centering
        \includegraphics[width=\textwidth]{image2.png}
        \caption{Подтверждение отправки данных}
    \end{subfigure}
    \caption{Основные панели веб-пакета IV-Conf для повседневной настройки.}
\end{figure}

\section{Примечания по использованию}
В большинстве случаев достаточно панелей конфигурации. Обновление прошивки выполняется отдельно в \autoref{ch:appendix}. Если параметр не применился, разорвите соединение MODBUS, подключитесь снова и убедитесь в правильности выбранного порта и скорости.
