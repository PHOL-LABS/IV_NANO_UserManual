\chapter{Конфигурирование через Android-бету IV-Conf}\label{ch:android-config}

В комплекте с индикаторами поставляется экспериментальное Android-приложение IV-Conf. Бета-версия позволяет прошивать пресеты и менять цветовые схемы, но для критически важных настроек PHOL-LABS рекомендует использовать веб-пакет.

\section{Подключение программатора}
\begin{enumerate}
    \item Вставьте внешний программатор в индикатор до подключения любых кабелей.
    \item Соедините программатор со смартфоном Android через кабель USB-C OTG. Если у телефона другой разъём, подключите USB-хаб или переходник USB-A между телефоном и программатором.
    \item Запустите бета-версию IV-Conf, подтвердите запрос на доступ к USB и нажмите \texttt{USB Connect}. Дождитесь статуса \texttt{Connected}.
\end{enumerate}

\section{Загрузка пресетов и режимов}
\begin{enumerate}
    \item Откройте боковое меню (значок из трёх полос) и выберите \texttt{Settings}.
    \item Выберите пресет, соответствующий установленному индикатору: Volts, Baro, Thermo, Lambda или Boost.
    \item Нажмите \texttt{Set parameters} и дождитесь завершения передачи. Индикатор должен оставаться неподвижным во время отправки данных.
    \item Используйте переключатели на экране, чтобы менять режимы точек или столбиков, а также слайдер яркости подсветки.
\end{enumerate}

\section{Редактирование градиентов}
\begin{enumerate}
    \item Выберите стиль градиента (по умолчанию: \texttt{Gradient}).
    \item Укажите количество сегментов (до четырёх) и задайте цвета и точки перехода для каждого сегмента.
    \item Нажмите \texttt{Set parameters}, чтобы записать обновлённый градиент в индикатор.
\end{enumerate}

\section{Обзор интерфейса}
\begin{table}[htbp]
    \centering
    \caption{Основные экраны подключения и запуска в Android-бете IV-Conf.}
    \label{tab:android-connection}
    \begin{tabular}{@{} c c c @{}}
        \subcaptionbox{Подключение программатора}{\includegraphics[width=0.3\textwidth]{image47.png}} &
        \subcaptionbox{Стартовый экран}{\includegraphics[width=0.3\textwidth]{image17.png}} &
        \subcaptionbox{Успешное USB-подключение}{\includegraphics[width=0.3\textwidth]{image18.png}} \\
    \end{tabular}
\end{table}

\begin{table}[htbp]
    \centering
    \caption{Настройки пресетов, режимов и градиентов в Android-бете IV-Conf.}
    \label{tab:android-presets}
    \begin{tabular}{@{} c c c @{}}
        \subcaptionbox{Выбор пресета}{\includegraphics[width=0.3\textwidth]{image19.png}} &
        \subcaptionbox{Настройка режима}{\includegraphics[width=0.3\textwidth]{image20.png}} &
        \subcaptionbox{Регулировка подсветки}{\includegraphics[width=0.3\textwidth]{image21.png}} \\
        \multicolumn{3}{c}{\subcaptionbox{Редактор градиента}{\includegraphics[width=0.3\textwidth]{image23.png}}}
    \end{tabular}
\end{table}

\section{Ограничения беты}
Бета-приложение активно дорабатывается. Возможны изменения интерфейса и эпизодические проблемы со стабильностью; при некорректном применении параметров вернитесь к веб-пакету. Сообщайте о воспроизводимых ошибках напрямую PHOL-LABS, чтобы обновления прошивки и приложения выходили синхронно.
