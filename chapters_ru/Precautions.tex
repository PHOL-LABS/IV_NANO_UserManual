\chapter{Запреты и меры предосторожности}\label{ch:precautions}

Программатор IV-Indicators Nano не является отдельным аксессуаром; это часть электроники индикатора. Неправильное обращение мгновенно выводит из строя каскад питания или замыкает открытые подпружиненные контакты. Всегда размещайте устройство на изолирующей поверхности и перед подачей питания соблюдайте правила ниже.

\section{Критически важные правила}
\begin{enumerate}
    \item \textbf{Никогда не подавайте питание на программатор отдельно от индикатора.} Подключение программатора к USB телефону или компьютеру без установленного индикатора сжигает электронику. Подавайте питание только после того, как программатор полностью установлен в корпус прибора.
    \item \textbf{Не вставляйте программатор в уже запитанный индикатор.} Все механические соединения выполняются при обесточенной системе. Сначала подключите программатор к индикатору, убедитесь, что он стоит ровно, и только потом подайте питание по USB.
    \item \textbf{Не подключайте и не отключайте программатор под напряжением.} Отсоединяйте USB-кабель перед извлечением программатора из индикатора и не допускайте перемещения во время передачи данных.
    \item \textbf{Фиксируйте индикатор во время программирования.} Качание на столе или свисающий кабель USB могут погнуть контакты и вызвать короткое замыкание. Надёжно зафиксируйте прибор, чтобы разъём не мог сместиться.
    \item \textbf{Избегайте проводящих поверхностей.} Корпуса ноутбуков, кузов автомобиля или открытые металлические столы замкнут контакты программатора, даже если устройство какое-то время работает. Всегда используйте чистый непроводящий коврик.
\end{enumerate}

\begin{quote}
    \textbf{Уведомление о гарантии.} Гарантия и возврат не предоставляются, если индикатор питался или программировался с нарушением этих запретов. PHOL-LABS Kft может предоставить замену, но она будет оплачена клиентом.
\end{quote}

\section{Неверные схемы подключения, которых следует избегать}
\begin{figure}[htbp]
    \centering
    \begin{subfigure}{0.45\textwidth}
        \centering
        \includegraphics[width=\textwidth]{image30.jpg}
        \caption{Не оставляйте программатор свисающим во время питания.}
    \end{subfigure}\hfill
    \begin{subfigure}{0.45\textwidth}
        \centering
        \includegraphics[width=\textwidth]{image28.png}
        \caption{Не размещайте контакты на проводящей поверхности.}
    \end{subfigure}
    \caption{Типичные случаи неправильного обращения, которые за секунды выводят электронику из строя.}
\end{figure}
