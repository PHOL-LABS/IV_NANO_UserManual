\chapter{Обзор продукта}\label{ch:product-description}

Модули IV-Indicators Nano — это компактные кнопочные индикаторы с единым корпусом и взаимозаменяемыми пресетами. Оборудование продаётся комплектом из трёх приборов или как отдельные индикаторы под конкретную задачу.

\section{Доступные варианты}
\begin{itemize}
    \item \textbf{Набор из трёх индикаторов IV-Indicators.} Полный комплект, включающий вольтметр, барометр и термометр для тех, кому нужен согласованный набор приборов.
    \item \textbf{IV-Indicators Voltmeter.} Поставляется с пресетами для контроля бортового напряжения в режимах столбиков или точек.
    \item \textbf{IV-Indicators Barometer.} Предназначен для мониторинга давления во впуске или топливной системе и откалиброван под датчик VAG 03C906051A.
    \item \textbf{IV-Indicators Thermometer.} Заводская калибровка рассчитана на терморезистор Ossca 01176, используемый в классических моделях Volkswagen.
    \item \textbf{IV-Indicators Lambda и Boost.} Специализированные индикаторы для контроля состава смеси и наддува в паре с рекомендованными датчиками и контроллерами.
\end{itemize}

\begin{figure}[htbp]
    \centering
    \begin{subfigure}{0.3\textwidth}
        \centering
        \includegraphics[width=\textwidth]{image33.jpg}
        \caption{Индикатор-вольтметр}
    \end{subfigure}\hfill
    \begin{subfigure}{0.3\textwidth}
        \centering
        \includegraphics[width=\textwidth]{image36.jpg}
        \caption{Индикатор-барометр}
    \end{subfigure}
    \par\medskip
    \begin{subfigure}{0.3\textwidth}
        \centering
        \includegraphics[width=\textwidth]{image35.jpg}
        \caption{Индикатор-термометр}
    \end{subfigure}\hfill
    \begin{subfigure}{0.3\textwidth}
        \centering
        \includegraphics[width=\textwidth]{image39.jpg}
        \caption{Варианты Lambda и Boost}
    \end{subfigure}
    \caption{Основные варианты IV-Indicators Nano, поставляемые PHOL-LABS Kft.}
\end{figure}

\section{Общие возможности}
Каждый индикатор можно перенастроить с помощью поставляемого программатора и программ IV-Conf. Пользователь может менять цветовые градиенты, задавать до четырёх переходов между сегментами и выбирать отображение точками или столбиками. Процедуры калибровки позволяют согласовать прибор с нештатными датчиками или настроить под пользовательские диапазоны напряжений. После настройки параметры сохраняются в устройстве, поэтому выбранный профиль остаётся даже после отключения питания.
