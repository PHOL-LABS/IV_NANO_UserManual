\chapter{Подготовка и установка}\label{ch:preparation}

Организуйте рабочее место до распаковки индикатора. Контакты пого должны оставаться чистыми и изолированными, а жгут — готовым для нужного датчика.

\section{Чек-лист рабочего места}
\begin{enumerate}
    \item Отсоедините аккумулятор автомобиля и разместите индикатор на непроводящей поверхности. Держите металлические корпуса ноутбуков и другие проводящие предметы подальше от контактов программатора.
    \item Примерьте место установки так, чтобы жгут без натяжения доставал до датчика. Прокладывайте кабели вдали от острых кромок и горячих частей двигателя.
    \item Установите разъём, входящий в комплект, и убедитесь, что программатор можно будет подключать позже без демонтажа окружающих панелей.
\end{enumerate}

\section{Подготовка жгута датчика}
\begin{itemize}
    \item Используйте подходящий разъём для выбранного пресета. Барометр требует вилку VAG \texttt{8K0973703}, а датчик наддува — герметичный трёхконтактный разъём, например Bosch \texttt{1928403966}.
    \item Для индикатора лямбда планируйте подключение двух датчиков кислорода: штатного узкополосного для ЭБУ и Bosch LSU~4.9, соединённого с отдельным контроллером. Аналоговый выход контроллера подайте в жгут индикатора.
    \item Обеспечьте разгрузку от натяжения и изоляцию каждого соединения. Контакты программатора и провода датчиков не рассчитаны на изгибы при включённом индикаторе.
\end{itemize}

\section{Поддержка}
PHOL-LABS Kft предоставляет однократную помощь по вопросам проводки и предлагает расширенные консультации на платной основе при необходимости адаптации нестандартных датчиков. Обратитесь в поддержку до подачи питания, если какой-либо этап подключения вызывает сомнения.
