\chapter{Technical specifications}\label{ch:technical-specifications}

\section{Unterstützte Sensoren}

\begin{table}[htbp]
    \centering
    \caption{Werkspresets und kompatible Sensoren}
    \label{tab:sensor-characteristics}

    \begin{tabularx}{\textwidth}{@{} l l l X @{}}
        \toprule
        Anzeige & Sensor & Versorgung & Hinweise \\
        \midrule
        Voltmeter & Fahrzeugkabelbaum & 12~V-System & Direkte Messung der Ladespannung. \\
        Barometer & VAG \texttt{03C906051A} & 5~V &
            10~bar Bereich, 0{,}4--0{,}5~V Nenn-Ausgang, Gewinde M10\,\texttimes\,1{,}0. \\
        Thermometer & Ossca \texttt{01176} NTC & Passiv &
            Referenzwerte in \autoref{app:temperature-table}. \\
        Lambda & Schmalband-O$_2$-Sensor & Passiv &
            0{,}1--0{,}9~V Hub, akzeptiert 0--5~V von Breitband-Controllern. \\
        Boost & Dacia \texttt{223657266R} & 5~V &
            20--250~kPa Bereich, 1{,}8~V Nennausgang, Einzel-Befestigungsbohrung. \\
        \bottomrule
    \end{tabularx}
\end{table}

\section{Sensor-Kompatibilitätstabellen}
Die folgenden Tabellen fassen kompatible Sensoren und Kalibrierreferenzen für die Werkspresets zusammen.

\subsection{Öl-/Kühlmitteltemperatursensoren}
\begin{table}[htbp]
    \centering
    \caption{Kompatibilität von Öl-/Kühlmitteltemperatursensoren}
    \label{tab:oil-coolant-sensors}
    \begin{tabularx}{\textwidth}{@{} l l X @{}}
        \toprule
        Kompatibler Sensor & Signal/Versorgung & Kalibrierparameter \\
        \midrule
        Ossca \texttt{01176} NTC-Thermistor (Thermometer-Preset) & Passiver NTC &
            Abgleich mit der Ossca-\texttt{01176}-Widerstandstabelle (z.\,B. 270{,}0~$\Omega$ bei 58~$^\circ$C bis 15{,}9~$\Omega$ bei 160~$^\circ$C). Nennkalibrierung: $R_{25} = 1$~k$\Omega$, $\beta = 3950$. Für kundenspezifische NTC-Sonden Beta und Nennwiderstand in IV-Conf anpassen. \\
        RIDEX \texttt{829S0003} (Opel/BMW/Volvo kompatibel) & 2-poliger NTC, Gewinde M12\,\texttimes\,1{,}5 &
            Nennwiderstand 2080~$\Omega$ bei 25~$^\circ$C und 294~$\Omega$ bei 80~$^\circ$C ($\beta \approx 3740$~K). Arbeitsbereich 25--80~$^\circ$C. Sensorgehäuse $\varnothing$~7{,}4~mm, Sechskant 19~mm, mit Dichtung geliefert. \\
        \bottomrule
    \end{tabularx}
\end{table}

\subsection{Lufttemperatursensoren}
\begin{table}[htbp]
    \centering
    \caption{Kompatibilität von Lufttemperatursensoren}
    \label{tab:air-temperature-sensors}
    \begin{tabularx}{\textwidth}{@{} l l X @{}}
        \toprule
        Kompatibler Sensor & Signal/Versorgung & Kalibrierparameter \\
        \midrule
        MFA-Außenlufttemperatursensor (VAG \texttt{171 919 379 A}, Golf Mk2/Jetta II/Passat B2/B3) & 2-adriger NTC, potentialfrei &
            Anzeigebereich ca. $-40^\circ$C bis $+96^\circ$C. Empirische Kalibrierung: $R_{25} \approx 510~\Omega$, $\beta \approx 3400$--3500~K (Abgleich zwischen $+4^\circ$C und $+50^\circ$C). Diagnose: 200~$\Omega \approx +50^\circ$C. \\
        NTC-Thermistor äquivalent zu Ossca \texttt{01176} (Thermometer-Preset) & Passiver NTC &
            Ossca-\texttt{01176}-Widerstandstabelle als Referenz verwenden und Beta/Nennwiderstand für kundenspezifische NTC-Sensoren über IV-Conf anpassen. \\
        \bottomrule
    \end{tabularx}
\end{table}

\subsection{Öldrucksensoren}
\begin{table}[htbp]
    \centering
    \caption{Kompatibilität von Öldrucksensoren}
    \label{tab:oil-pressure-sensors}
    \begin{tabularx}{\textwidth}{@{} l l X @{}}
        \toprule
        Kompatibler Sensor & Signal/Versorgung & Kalibrierparameter \\
        \midrule
        VAG \texttt{03C906051A} Drucksensor (Barometer-Preset) & 5~V-Versorgung, Analogausgang &
            10~bar absoluter Bereich, 0{,}4--0{,}5~V Nennausgang im Ruhezustand. Messbereich mit dem Barometer-Preset konfigurieren und min./max.-Bereiche bei Bedarf in IV-Conf anpassen. \\
        \bottomrule
    \end{tabularx}
\end{table}

\subsection{Kraftstoffstandgeber}
\begin{table}[htbp]
    \centering
    \caption{Kompatibilität von Kraftstoffstandsensoren}
    \label{tab:fuel-level-sensors}
    \begin{tabularx}{\textwidth}{@{} l l X @{}}
        \toprule
        Kompatibler Sensor & Signal/Versorgung & Kalibrierparameter \\
        \midrule
        Kundenspezifischer Sensor (kein bestimmtes Modell genannt) & Abhängig vom Sensor &
            Min./Max.-Messbereich in IV-Conf auf den installierten Sensor abstimmen. \\
        \bottomrule
    \end{tabularx}
\end{table}

\subsection{Ladedrucksensoren}
\begin{table}[htbp]
    \centering
    \caption{Kompatibilität von Ladedrucksensoren}
    \label{tab:boost-pressure-sensors}
    \begin{tabularx}{\textwidth}{@{} l l X @{}}
        \toprule
        Kompatibler Sensor & Signal/Versorgung & Kalibrierparameter \\
        \midrule
        Dacia \texttt{223657266R} (auch \texttt{161B0004}/\texttt{8200225971}) & 5~V-Versorgung, Analogausgang &
            Arbeitsbereich 20--250~kPa, $\sim$1{,}8~V Nennausgang bei Ladedruck. Boost-Preset verwenden und Range min/max bei Bedarf anpassen. \\
        \bottomrule
    \end{tabularx}
\end{table}

\subsection{Breitband-Lambda}
\begin{table}[htbp]
    \centering
    \caption{Kompatibilität von Breitband-Lambda-Controllern}
    \label{tab:wideband-lambda}
    \begin{tabularx}{\textwidth}{@{} l l X @{}}
        \toprule
        Kompatibler Sensor & Signal/Versorgung & Kalibrierparameter \\
        \midrule
        Breitband-Controller (SLC Free, DIY-EFI TinyWB, Sigma Lambda Controller Free~2) mit Bosch LSU~4.9-Sonde & Lineares 0--5~V-Ausgangssignal zur Anzeige &
            Breitband-Controller wandeln die LSU~4.9-Sonde in ein lineares 0--5~V-Signal für die Anzeige; Schmalband-Sensoren liefern einen 0{,}1--0{,}9~V-Hub. \\
        \bottomrule
    \end{tabularx}
\end{table}

\section{Barometer-Preset}
Die Barometer-Anzeige ist für den Drucksensor \texttt{03C906051A} des Volkswagen-Konzerns ausgelegt.
\begin{itemize}
    \item Spannungsversorgung: 5~V geregelt.
    \item Messbereich: 10~bar absolut.
    \item Nenn-Ausgangsspannung: 0{,}4--0{,}5~V im Ruhezustand.
    \item Mechanisches Gewinde: M10\,\texttimes\,1{,}0.
    \item Stecker: VAG \texttt{8K0973703}.
\end{itemize}

\section{Thermometer-Preset}
Thermometer-Presets werden für den NTC-Thermistor Ossca\,01176 kalibriert ausgeliefert, der häufig in Golf~II-Kühlsystemen verbaut ist. Verwenden Sie die Widerstandstabelle in \autoref{app:temperature-table}, um alternative Sonden zu überprüfen. Für kundenspezifische NTC-Sensoren passen Sie den Beta-Koeffizienten und den Nennwiderstand über die IV-Conf-Werkzeuge an.

\section{Boost-Preset}
Die Boost-Anzeige erwartet den Sensor Dacia \texttt{223657266R} (\texttt{161B0004}/\texttt{8200225971}).
\begin{itemize}
    \item Dreipoliger Stecker (verwenden Sie einen abgedichteten Bosch \texttt{1928403966} oder ein Äquivalent).
    \item Versorgungsspannung: +5~V.
    \item Arbeitsbereich: 20--250~kPa.
    \item Analoger Ausgang: etwa +1{,}8~V bei nominalem Ladedruck.
\end{itemize}

\section{Lambda-Integration}
IV-Indicators Lambda liest Schmalband-Sonden direkt aus, ist bei getunten Motoren jedoch für den Betrieb mit einem Breitband-Controller gedacht. Empfohlene Controller sind SLC Free, DIY-EFI TinyWB und Sigma Lambda Controller Free~2. Jedes Gerät wandelt eine Bosch LSU~4.9-Sonde in ein lineares 0--5~V-Signal um, das die Anzeige darstellen kann.

\section{Sensorreferenzen}
\begin{figure}[htbp]
    \centering
    \begin{subfigure}{0.3\textwidth}
        \centering
        \includegraphics[width=\textwidth]{image41.jpg}
        \caption{VAG 03C906051A Drucksensor}
    \end{subfigure}\hfill
    \begin{subfigure}{0.3\textwidth}
        \centering
        \includegraphics[width=\textwidth]{image9.png}
        \caption{Pinbelegung für den Barometer-Kabelbaum}
    \end{subfigure}\hfill
    \begin{subfigure}{0.3\textwidth}
        \centering
        \includegraphics[width=\textwidth]{image42.jpg}
        \caption{Beispielanschluss des Boost-Sensors}
    \end{subfigure}
    \caption{Referenzhardware für die Barometer- und Boost-Presets.}
\end{figure}
