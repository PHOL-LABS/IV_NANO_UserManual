\chapter{Technical specifications}\label{ch:technical-specifications}

\section{Unterstützte Sensoren}

\begin{table}[htbp]
    \centering
    \caption{Werkspresets und kompatible Sensoren}
    \label{tab:sensor-characteristics}

    \begin{tabularx}{\textwidth}{@{} l l l X @{}}
        \toprule
        Anzeige & Sensor & Versorgung & Hinweise \\
        \midrule
        Voltmeter & Fahrzeugkabelbaum & 12~V-System & Direkte Messung der Ladespannung. \\
        Barometer & VAG \texttt{03C906051A} & 5~V &
            10~bar Bereich, 0{,}4--0{,}5~V Nenn-Ausgang, Gewinde M10\,\texttimes\,1{,}0. \\
        Thermometer & Ossca \texttt{01176} NTC & Passiv &
            Referenzwerte in \autoref{app:temperature-table}. \\
        Lambda & Schmalband-O$_2$-Sensor & Passiv &
            0{,}1--0{,}9~V Hub, akzeptiert 0--5~V von Breitband-Controllern. \\
        Boost & Dacia \texttt{223657266R} & 5~V &
            20--250~kPa Bereich, 1{,}8~V Nennausgang, Einzel-Befestigungsbohrung. \\
        \bottomrule
    \end{tabularx}
\end{table}

\section{Barometer-Preset}
Die Barometer-Anzeige ist für den Drucksensor \texttt{03C906051A} des Volkswagen-Konzerns ausgelegt.
\begin{itemize}
    \item Spannungsversorgung: 5~V geregelt.
    \item Messbereich: 10~bar absolut.
    \item Nenn-Ausgangsspannung: 0{,}4--0{,}5~V im Ruhezustand.
    \item Mechanisches Gewinde: M10\,\texttimes\,1{,}0.
    \item Stecker: VAG \texttt{8K0973703}.
\end{itemize}

\section{Thermometer-Preset}
Thermometer-Presets werden für den NTC-Thermistor Ossca\,01176 kalibriert ausgeliefert, der häufig in Golf~II-Kühlsystemen verbaut ist. Verwenden Sie die Widerstandstabelle in \autoref{app:temperature-table}, um alternative Sonden zu überprüfen. Für kundenspezifische NTC-Sensoren passen Sie den Beta-Koeffizienten und den Nennwiderstand über die IV-Conf-Werkzeuge an.

\section{Boost-Preset}
Die Boost-Anzeige erwartet den Sensor Dacia \texttt{223657266R} (\texttt{161B0004}/\texttt{8200225971}).
\begin{itemize}
    \item Dreipoliger Stecker (verwenden Sie einen abgedichteten Bosch \texttt{1928403966} oder ein Äquivalent).
    \item Versorgungsspannung: +5~V.
    \item Arbeitsbereich: 20--250~kPa.
    \item Analoger Ausgang: etwa +1{,}8~V bei nominalem Ladedruck.
\end{itemize}

\section{Lambda-Integration}
IV-Indicators Lambda liest Schmalband-Sonden direkt aus, ist bei getunten Motoren jedoch für den Betrieb mit einem Breitband-Controller gedacht. Empfohlene Controller sind SLC Free, DIY-EFI TinyWB und Sigma Lambda Controller Free~2. Jedes Gerät wandelt eine Bosch LSU~4.9-Sonde in ein lineares 0--5~V-Signal um, das die Anzeige darstellen kann.

\section{Sensorreferenzen}
\begin{figure}[htbp]
    \centering
    \begin{subfigure}{0.3\textwidth}
        \centering
        \includegraphics[width=\textwidth]{image41.jpg}
        \caption{VAG 03C906051A Drucksensor}
    \end{subfigure}\hfill
    \begin{subfigure}{0.3\textwidth}
        \centering
        \includegraphics[width=\textwidth]{image9.png}
        \caption{Pinbelegung für den Barometer-Kabelbaum}
    \end{subfigure}\hfill
    \begin{subfigure}{0.3\textwidth}
        \centering
        \includegraphics[width=\textwidth]{image42.jpg}
        \caption{Beispielanschluss des Boost-Sensors}
    \end{subfigure}
    \caption{Referenzhardware für die Barometer- und Boost-Presets.}
\end{figure}
