\chapter{Preparation and installation}\label{ch:preparation}

Richten Sie den Arbeitsplatz ein, bevor Sie die Anzeige auspacken. Der Pogo-Stecker muss sauber und isoliert bleiben, und der Kabelbaum sollte für den benötigten Sensor bereitliegen.

\section{Checkliste für den Arbeitsplatz}
\begin{enumerate}
    \item Trennen Sie die Fahrzeugbatterie und legen Sie die Anzeige auf eine nicht leitende Unterlage. Halten Sie metallische Laptop-Gehäuse und andere leitende Gegenstände von den Pogo-Pins fern.
    \item Probieren Sie den Montageort aus, damit der Kabelbaum den Zielsensor ohne Zugbelastung erreicht. Führen Sie Kabel an scharfen Kanten oder heißen Motorteilen vorbei.
    \item Montieren Sie die mitgelieferte Fassung und stellen Sie sicher, dass der Programmer später angebracht werden kann, ohne angrenzende Verkleidungsteile entfernen zu müssen.
\end{enumerate}

\section{Vorbereitung des Sensorkabelbaums}
\begin{itemize}
    \item Verwenden Sie den passenden Stecker für das ausgewählte Preset. Das Barometer nutzt einen VAG-Stecker \texttt{8K0973703}, während der Boost-Sensor einen abgedichteten dreipoligen Stecker wie den Bosch \texttt{1928403966} benötigt.
    \item Planen Sie bei einer Lambda-Anzeige mit zwei Sauerstoffsonden: die serienmäßige Schmalband-Sonde für das Motorsteuergerät und eine Bosch LSU~4.9, die mit einem eigenen Steuergerät verbunden ist. Führen Sie den analogen Ausgang des Controllers in den Kabelbaum der Anzeige.
    \item Sorgen Sie bei jeder Verbindung für Zugentlastung und Isolierung. Der Pogo-Stecker und die Sensorleitungen sind nicht dafür ausgelegt, sich zu bewegen, während die Anzeige unter Strom steht.
\end{itemize}

\section{Support}
PHOL-LABS Kft bietet eine einmalige Unterstützungssitzung für Verdrahtungsfragen und stellt kostenpflichtige, verlängerte Beratungen bereit, wenn spezielle Sensormodifikationen nötig sind. Wenden Sie sich vor der Inbetriebnahme an den Support, wenn ein Verdrahtungsschritt unklar ist.
