\chapter{Product overview}\label{ch:product-description}

Die Module der IV-Indicators Nano sind kompakte Taster-Instrumente mit gemeinsamem Gehäuse und austauschbaren Presets. Die Hardware wird als gebündeltes Dreier-Set oder als dedizierte Anzeige für eine einzelne Messaufgabe angeboten.

\section{Verfügbare Varianten}
\begin{itemize}
    \item \textbf{IV-Indicators Dreier-Set.} Ein komplettes Paket mit Volt-, Barometer- und Thermometer-Anzeige für Fahrer, die ein aufeinander abgestimmtes Set wünschen.
    \item \textbf{IV-Indicators Voltmeter.} Wird mit Presets ausgeliefert, die die Bordspannung des Fahrzeugs wahlweise als Balken oder Punkt anzeigen.
    \item \textbf{IV-Indicators Barometer.} Für die Überwachung von Saugrohr- oder Kraftstoffdruck mit Profilen, die auf den VAG-Sensor 03C906051A abgestimmt sind.
    \item \textbf{IV-Indicators Thermometer.} Werksseitig kalibriert auf den in klassischen Volkswagen verbauten NTC-Temperatursensor Ossca 01176.
    \item \textbf{IV-Indicators Lambda und Boost.} Spezialisierte Anzeigen für Luft-Kraftstoff-Verhältnis und Ladedruck, wenn sie mit den empfohlenen Sensoren und Steuergeräten kombiniert werden.
\end{itemize}

\begin{figure}[htbp]
    \centering
    \begin{subfigure}{0.3\textwidth}
        \centering
        \includegraphics[width=\textwidth]{image33.jpg}
        \caption{Voltmeter-Anzeige}
    \end{subfigure}\hfill
    \begin{subfigure}{0.3\textwidth}
        \centering
        \includegraphics[width=\textwidth]{image36.jpg}
        \caption{Barometer-Anzeige}
    \end{subfigure}
    \par\medskip
    \begin{subfigure}{0.3\textwidth}
        \centering
        \includegraphics[width=\textwidth]{image35.jpg}
        \caption{Thermometer-Anzeige}
    \end{subfigure}\hfill
    \begin{subfigure}{0.3\textwidth}
        \centering
        \includegraphics[width=\textwidth]{image39.jpg}
        \caption{Lambda- und Boost-Varianten}
    \end{subfigure}
    \caption{Zentrale Produktvarianten der IV-Indicators Nano, geliefert von PHOL-LABS Kft.}
\end{figure}

\section{Gemeinsame Funktionen}
Jede Anzeige lässt sich mit dem mitgelieferten Programmer und der IV-Conf-Software neu konfigurieren. Besitzer können Farbverläufe anpassen, bis zu vier Segmentübergänge definieren und zwischen Punkt- oder Balkendarstellung wählen. Kalibrierungsroutinen erlauben die Anpassung an Zubehörsensoren oder benutzerdefinierte Spannungsbereiche. Nach der Konfiguration speichert das Gerät die Einstellungen, sodass das gewählte Profil auch ohne Versorgung erhalten bleibt.
