\chapter{Prohibitions and precautions}\label{ch:precautions}

Der Programmer der IV-Indicators Nano ist kein eigenständiges Zubehör, sondern Teil der Instrumentenelektronik. Unsachgemäße Handhabung kann sofort die Spannungsregelung zerstören oder die freiliegenden Pogo-Pins kurzschließen. Stellen Sie das Gerät immer auf eine isolierende Unterlage und befolgen Sie die folgenden Regeln, bevor Sie Strom anlegen.

\section{Kritische Regeln}
\begin{enumerate}
    \item \textbf{Den Programmer niemals außerhalb der Anzeige mit Strom versorgen.} Das Anschließen des Programmers an ein USB-Telefon oder einen Computer ohne montierte Anzeige zerstört die Elektronik. Versorgen Sie das Werkzeug erst, wenn es vollständig im Gehäuse der Anzeige sitzt.
    \item \textbf{Stecken Sie den Programmer nicht in eine stromführende Anzeige.} Stellen Sie jede mechanische Verbindung im spannungslosen Zustand her. Stecken Sie den Programmer zuerst in die Anzeige, prüfen Sie den geraden Sitz und verbinden Sie erst danach die USB-Versorgung.
    \item \textbf{Verbinden oder trennen Sie den Programmer nicht, solange er unter Strom steht.} Trennen Sie vor dem Entfernen des Programmers von der Anzeige das USB-Kabel und vermeiden Sie Bewegungen während der Datenübertragung.
    \item \textbf{Sichern Sie die Anzeige während des Programmierens.} Wackeln auf dem Tisch oder Hängen am USB-Kabel kann Pogo-Pins verbiegen und Kurzschlüsse verursachen. Stützen Sie das Gerät so ab, dass sich der Stecker nicht bewegen kann.
    \item \textbf{Vermeiden Sie leitende Oberflächen.} Laptopgehäuse, Karosserieteile oder blanke Metallwerkbänke kurzschließen die Programmierkontakte, auch wenn das System zunächst zu funktionieren scheint. Verwenden Sie stets eine saubere, nicht leitende Unterlage.
\end{enumerate}

\begin{quote}
    \textbf{Garantiehinweis.} Es besteht keine Garantie oder Rücknahme, wenn die Anzeige entgegen diesen Verboten mit Strom versorgt oder programmiert wurde. PHOL-LABS Kft kann Ersatz liefern, dieser wird dem Kunden jedoch berechnet.
\end{quote}

\section{Fehlerhafte Aufbauten, die zu vermeiden sind}
\begin{figure}[htbp]
    \centering
    \begin{subfigure}{0.45\textwidth}
        \centering
        \includegraphics[width=\textwidth]{image30.jpg}
        \caption{Lassen Sie den Programmer nicht im Betrieb hängen.}
    \end{subfigure}\hfill
    \begin{subfigure}{0.45\textwidth}
        \centering
        \includegraphics[width=\textwidth]{image28.png}
        \caption{Vermeiden Sie leitende Arbeitsflächen unter den Pogo-Pins.}
    \end{subfigure}
    \caption{Typische Fehlbedienungen, die die Elektronik innerhalb von Sekunden zerstören können.}
\end{figure}
