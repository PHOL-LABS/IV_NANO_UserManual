\chapter{Sensor connections}\label{ch:connections}

This chapter summarises the recommended wiring practices for each indicator variant. Detailed diagrams will be added in future revisions; follow the sensor datasheets and vehicle wiring standards in the meantime.

\section{Voltage monitoring}
Connect the voltmeter indicator across the vehicle's switched 12~V supply and ground. Ensure the harness includes appropriate fusing and that the indicator shares a common ground reference with other instrumentation.

\section{Oil pressure (barometer)}
The barometer indicator uses the VAG \texttt{03C906051A} sensor. Supply it with 5~V and connect the analogue output to the signal input on the indicator harness. Use the \texttt{8K0973703} connector or an equivalent sealed plug.

\section{Temperature monitoring}
The thermometer indicator pairs with an NTC sensor such as Ossca \texttt{01176}. Route both leads to the indicator harness and verify continuity before powering the device. Use \autoref{app:temperature-table} to confirm the sensor is within tolerance.

\section{Lambda monitoring}
Narrowband lambda sensors output 0.1--0.9~V swings that can be fed directly into the indicator. Wideband systems require a dedicated controller (for example SLC Free, DIY-EFI TinyWB, or Sigma Lambda Controller Free~2) that exposes a 0--5~V analogue output. Connect the controller output to the indicator signal input and common ground.

\section{Boost monitoring}
The boost indicator expects a Dacia \texttt{223657266R} pressure sensor powered at 5~V. Use a sealed three-pin connector such as Bosch \texttt{1928403966} when the OEM connector is unavailable. Mount the sensor securely and keep the hose length short to minimise response lag.
