\chapter{Operating principle}\label{ch:operating-principle}

All IV-Indicators Nano variants share a common architecture. The indicator body houses the display, colour-controllable backlight, and an embedded controller that exposes a four-pin programming port. The supplied programmer bridges this port to USB for Android phones or computers, allowing the firmware to receive configuration updates, presets, and firmware images.

Each indicator ships with a default preset for its intended sensor:

\begin{itemize}
    \item \textbf{Voltmeter} --- measures system voltage directly from the vehicle harness.
    \item \textbf{Barometer} --- expects a VAG pressure sensor (\texttt{03C906051A}) powered at 5~V with a 0.4--0.5~V nominal output.
    \item \textbf{Thermometer} --- reads NTC thermistors compatible with Ossca \texttt{01176}; resistance values are listed in \autoref{app:temperature-table}.
    \item \textbf{Lambda} --- supports narrowband oxygen sensors out of the box and accepts 0--5~V analogue signals from wideband controllers.
    \item \textbf{Boost} --- interfaces with the Dacia \texttt{223657266R} boost pressure sensor powered at 5~V.
\end{itemize}

The lambda indicator can be paired with dedicated wideband controllers (for example SLC Free, DIY-EFI TinyWB, or Sigma Lambda Controller Free~2) that translate the Bosch LSU~4.9 sensor output to an analogue voltage. Connect the controller to the LSU~4.9 according to its manual and route the linear analogue output (typically 0--5~V or 0.5--4.5~V) to the IV-Indicators Lambda input.

Sockets differ across sensors. The barometer uses the VAG \texttt{8K0973703} connector, while the recommended boost sensor requires a generic three-pin connector such as Bosch \texttt{1928403966}. When an official connector is unavailable, use sealed automotive-grade plugs and ensure strain relief on the harness.
