\chapter{Operating principle}\label{ch:operating-principle}

Every IV-Indicators Nano gauge integrates a segmented display, RGB backlight, and a microcontroller that stores measurement presets. A four-pin pogo connector on the rear mates with the supplied programmer so the indicator can receive firmware updates and configuration profiles from an Android phone or desktop browser.

\section{Measurement presets}
Each indicator ships with a preset aligned to its intended role:
\begin{itemize}
    \item \textbf{Voltmeter} --- monitors system voltage directly from the vehicle harness.
    \item \textbf{Barometer} --- targets the VAG 03C906051A pressure sensor running from a regulated 5~V supply.
    \item \textbf{Thermometer} --- reads NTC thermistors equivalent to the Ossca\,01176 probe listed in \autoref{app:temperature-table}.
    \item \textbf{Lambda} --- interprets narrowband oxygen sensor signals and accepts linear 0--5~V outputs from wideband controllers.
    \item \textbf{Boost} --- interfaces with the Dacia 223657266R sensor\\ (also known as 161B0004/8200225971).
\end{itemize}

\section{Configuration workflow}
Configuration data is pushed over USB by the IV-Conf applications. The Android beta and the web pack both expose colour, gradient, and mode controls so the same hardware can render dot or bar indicators. You can define up to four colour segments, tune backlight intensity, and adjust minimum/maximum measurement ranges to match your sensors. Calibrated values are retained even after the gauge is unplugged, making it easy to experiment without losing a preferred setup.

\section{Sensor integration}
The barometer preset expects the OEM VAG \texttt{8K0973703} connector. The boost indicator is supplied without a dedicated harness; a sealed three-pin automotive connector such as the Bosch \texttt{1928403966} is recommended. For lambda monitoring, PHOL-LABS advises pairing the gauge with an external wideband controller (SLC Free, DIY-EFI TinyWB, Sigma Lambda Controller Free~2, or similar) that conditions a Bosch LSU~4.9 probe into a stable analogue voltage. Route the controller's linear output to the gauge input while following the controller manual for heater power and calibration.
