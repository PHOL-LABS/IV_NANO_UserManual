\chapter{Prohibitions and precautions}\label{ch:precautions}

IV-Indicators Nano modules share a common programmer and enclosure design. Misusing the power connection or attaching the programmer while the indicator is live can permanently damage the electronics. Please read the following rules before unpacking the hardware:

\begin{enumerate}
    \item \textbf{Never power the programmer outside the indicator.} The programmer is an integral part of the gauge. Plugging it into USB without the indicator connected can destroy the electronics.
    \item \textbf{Do not insert the programmer into a powered indicator.} Always make the mechanical connection first, ensure the programmer and indicator sit firmly in place, and only then apply power.
    \item \textbf{Do not attach or detach the programmer while it is powered.} Disconnect the USB cable from the phone or computer before unplugging the programmer from the gauge.
    \item \textbf{Keep the device stable during programming.} Movement can bend the pogo pins and cause shorts. Place the indicator on a non-conductive surface before flashing or configuring it.
    \item \textbf{Avoid conductive work surfaces.} Laptops, tabletops, or vehicle bodywork can short exposed contacts. Use a clean insulating mat whenever the rear of the indicator is exposed.
    \item \textbf{Do not rely on temporary success.} The gauge may appear to work when placed on metal, but even a slight movement can short the pins and blow the electronics. Treat this as permanent damage---warranty coverage is void in such cases.
\end{enumerate}

\begin{dangerbox}
    There is no warranty or returns if the device has been powered or programmed in a way that contradicts the rules above. Paid replacements are available from PHOL-LABS Kft if damage occurs.
\end{dangerbox}
