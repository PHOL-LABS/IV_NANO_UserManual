\chapter{Typical setup cases}\label{ch:setup-cases}

This section summarises common configuration tasks handled through the IV-Conf web pack or Android app. Always connect to the correct serial port and confirm that \texttt{Connect MODBUS} reports a successful link before applying changes.

\section{Adjusting colours}
\begin{enumerate}
    \item Press \texttt{Get data} to read the current configuration.
    \item Modify the hue or gradient settings as required. Text colour previews match the indicator output.
    \item Adjust the backlight parameters (labelled ``Bl'') to set the background intensity.
    \item Press \texttt{Send data} and wait until the transfer completes.
\end{enumerate}

\section{Correcting measurement ranges}
For voltmeter, barometer, boost, or lambda indicators:
\begin{enumerate}
    \item Press \texttt{Get data} to read the configuration.
    \item Update \texttt{Range min} and \texttt{Range max} to match the desired display range.
    \item Press \texttt{Send data} and allow 10--15 seconds for the upload to finish.
\end{enumerate}

\section{Thermometer calibration}
\begin{enumerate}
    \item Verify the sensor resistance with a multimeter and compare it to the reference values in \autoref{app:temperature-table}. Replace the sensor if the measurement deviates significantly.
    \item Press \texttt{Get data} and update the beta coefficient and nominal resistance fields if a custom NTC sensor is used.
    \item Press \texttt{Send data} to apply the new parameters.
\end{enumerate}

\section{Switching between bar and dot display}
\begin{enumerate}
    \item Press \texttt{Get data}.
    \item Set the \texttt{Bar mode} flag to 1 for bar display or clear it for dot display.
    \item Press \texttt{Send data}.
\end{enumerate}
