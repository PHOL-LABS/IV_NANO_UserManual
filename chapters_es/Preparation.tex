\chapter{Preparaci\'on e instalaci\'on}\label{ch:preparation}

Prepare el espacio de trabajo antes de desempacar el indicador. El conector pogo debe permanecer limpio y aislado, y el mazo debe estar listo para el sensor necesario.

\section{Lista de verificaci\'on del lugar de trabajo}
\begin{enumerate}
    \item Desconecte la bater\'ia del veh\'iculo y coloque el indicador sobre una superficie no conductora. Mantenga alejadas las carcasas met\'alicas de port\'atiles y otros objetos conductores de los pines pogo.
    \item Pruebe la ubicaci\'on de montaje para que el mazo pueda alcanzar el sensor objetivo sin tensiones. Gu\'ie los cables lejos de bordes afilados o partes del motor de alta temperatura.
    \item Instale el z\'ocalo suministrado con el kit y confirme que el programador pueda conectarse m\'as adelante sin retirar los embellecedores cercanos.
\end{enumerate}

\section{Preparaci\'on del mazo de sensores}
\begin{itemize}
    \item Utilice el conector correcto para el preajuste seleccionado. El bar\'ometro depende de un enchufe VAG \texttt{8K0973703}, mientras que el sensor de turbo necesita un conector sellado de tres pines como el Bosch \texttt{1928403966}.
    \item Al instalar un indicador lambda, planifique dos sensores de ox\'igeno: el sensor de banda estrecha de serie para la ECU y un Bosch LSU~4.9 conectado a un controlador dedicado. Conduzca la salida anal\'ogica del controlador hacia el mazo del indicador.
    \item Proporcione alivio de tensi\'on y aislamiento en cada empalme. El conector pogo y los cables del sensor no est\'an dise\~nados para flexionarse mientras el indicador est\'a alimentado.
\end{itemize}

\section{Soporte}
PHOL-LABS Kft ofrece una sesi\'on de asistencia \'unica para dudas de cableado y proporciona consultas ampliadas de pago cuando se requieren adaptaciones de sensores personalizadas. P\'ongase en contacto con soporte antes de alimentar el sistema si alguna etapa de cableado no est\'a clara.
