\chapter{Függelék}\label{ch:appendix}

\section{IV-Conf webes csomag használata firmware-frissítéshez}\label{app:reflashing}
Firmware frissítéséhez használja a webes csomagot. Az alábbi lépéseket áramtalanítás nélkül végezze el, amint az átvitel elindult.
\begin{enumerate}
    \item Nyissa meg a \url{https://phol-labs.com/iv} oldalt, és csatlakozzon a programozóhoz a \autoref{ch:web-config} fejezetben leírtak szerint.
    \item Győződjön meg róla, hogy a baud ráta 115200, majd kattintson a \texttt{Connect MODBUS} gombra. Várja meg a zöld státusz üzenetet.
    \item Nyomja meg a \texttt{Stop Send Data} gombot, hogy leállítsa az automatikus előbeállítás-feltöltést.
    \item Kattintson a firmware ikonra, válassza ki az indikátorhoz kapott frissítőcsomagot, majd erősítse meg a kérést.
    \item Hagyja érintetlenül az indikátort, amíg a folyamatjelző végigfut és az eszköz újraindul.
\end{enumerate}

\begin{figure}[htbp]
    \centering
    \begin{subfigure}{0.3\textwidth}
        \centering
        \includegraphics[width=\textwidth]{image15.png}
        \caption{A webes csomag megnyitása}
    \end{subfigure}\hfill
    \begin{subfigure}{0.3\textwidth}
        \centering
        \includegraphics[width=\textwidth]{image11.png}
        \caption{Soros port kiválasztása}
    \end{subfigure}\hfill
    \begin{subfigure}{0.3\textwidth}
        \centering
        \includegraphics[width=\textwidth]{image1.png}
        \caption{Hozzáférés engedélyezése}
    \end{subfigure}
    \par\medskip
    \begin{subfigure}{0.3\textwidth}
        \centering
        \includegraphics[width=\textwidth]{image8.png}
        \caption{115200 baud beállítása}
    \end{subfigure}\hfill
    \begin{subfigure}{0.3\textwidth}
        \centering
        \includegraphics[width=\textwidth]{image25.png}
        \caption{Aktív átvitelek leállítása}
    \end{subfigure}\hfill
    \begin{subfigure}{0.3\textwidth}
        \centering
        \includegraphics[width=\textwidth]{image14.png}
        \caption{MODBUS kapcsolat}
    \end{subfigure}
    \par\medskip
    \begin{subfigure}{0.3\textwidth}
        \centering
        \includegraphics[width=\textwidth]{image6.png}
        \caption{Firmware párbeszéd megnyitása}
    \end{subfigure}\hfill
    \begin{subfigure}{0.3\textwidth}
        \centering
        \includegraphics[width=\textwidth]{image46.png}
        \caption{Frissítőcsomag kiválasztása}
    \end{subfigure}\hfill
    \begin{subfigure}{0.3\textwidth}
        \centering
        \includegraphics[width=\textwidth]{image7.png}
        \caption{Feltöltés megerősítése}
    \end{subfigure}
    \par\medskip
    \begin{subfigure}{0.3\textwidth}
        \centering
        \includegraphics[width=\textwidth]{image12.png}
        \caption{Várakozás a befejezésre}
    \end{subfigure}
    \caption{Az IV-Conf webes csomag képernyői a firmware újraírása során.}
\end{figure}

\section{Hőmérséklet-érzékelő referencia}\label{app:temperature-table}
Az alábbi táblázat az Ossca \texttt{01176} NTC érzékelő névleges ellenállásértékeit sorolja fel, amelyet a hőmérő előbeállítás használ.

\begin{table}[htbp]
    \centering
    \caption{NTC ellenállás referencia}
    \label{tab:ntc-reference}
    \begin{tabular}{@{} rr @{}}
        \toprule
        Ellenállás (\si{\ohm}) & Hőmérséklet (\si{\celsius}) \\
        \midrule
        270.0 & 58 \\
        220.0 & 63 \\
        199.8 & 66 \\
        111.0 & 83 \\
        73.8 & 98 \\
        55.0 & 108 \\
        48.8 & 113 \\
        44.0 & 117 \\
        37.2 & 124 \\
        32.1 & 130 \\
        28.2 & 136 \\
        25.1 & 141 \\
        22.7 & 146 \\
        20.4 & 151 \\
        19.0 & 155 \\
        18.8 & 155 \\
        15.9 & 160 \\
        \bottomrule
    \end{tabular}
\end{table}

A függelék későbbi kiadásai további érzékelőadatokat és bekötési diagramokat fognak tartalmazni.
