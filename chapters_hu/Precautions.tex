\chapter{Tiltások és óvintézkedések}\label{ch:precautions}

Az IV-Indicators Nano programozó nem önálló kiegészítő; az indikátor elektronikájának része. A helytelen kezelés azonnal tönkreteheti a tápellátó fokozatot vagy rövidre zárhatja a szabadon lévő pogo pineket. Mindig szigetelő felületre helyezze a készüléket, és kövesse az alábbi szabályokat, mielőtt áramot adna rá.

\section{Kritikus szabályok}
\begin{enumerate}
    \item \textbf{Soha ne helyezze feszültség alá a programozót az indikátortól külön.} Ha a programozót a műszer nélkül csatlakoztatja telefonhoz vagy számítógéphez, az elektronikát kiégeti. Csak akkor adjon tápot, ha a programozó teljesen illeszkedik az indikátorházba.
    \item \textbf{Ne helyezze be a programozót feszültség alatt lévő indikátorba.} Minden mechanikai csatlakozást áramtalanított állapotban végezzen. Először csatlakoztassa a programozót az indikátorhoz, ellenőrizze, hogy egyenesen áll-e, és csak ezután kösse rá az USB-tápot.
    \item \textbf{Ne csatlakoztassa vagy válassza le a programozót, amíg áram alatt van.} Az indikátorról való eltávolítás előtt húzza ki az USB-kábelt, és adatátvitel közben minimalizálja a mozgatást.
    \item \textbf{Rögzítse a műszert programozás közben.} Az asztalon billegő vagy a kábelen lógó eszköz elhajlíthatja a pogo pineket és rövidzárt okozhat. Támasztja alá úgy, hogy a csatlakozó ne mozdulhasson el.
    \item \textbf{Kerülje a vezető felületeket.} Laptopházak, karosszériaelemek vagy nyers fém asztalok rövidre zárják a programozó érintkezőit, még ha a rendszer átmenetileg működőképesnek tűnik is. Mindig tiszta, szigetelő alátétet használjon.
\end{enumerate}

\begin{quote}
    \textbf{Garanciális figyelmeztetés.} Nem jár garancia vagy visszavétel, ha az indikátort a fenti tiltások figyelmen kívül hagyásával helyezték feszültség alá vagy programozták. A PHOL-LABS Kft tud pótlást biztosítani, de ennek költsége az ügyfelet terheli.
\end{quote}

\section{Hibás összeállítások, amelyeket kerülni kell}
\begin{figure}[htbp]
    \centering
    \begin{subfigure}{0.45\textwidth}
        \centering
        \includegraphics[width=\textwidth]{image30.jpg}
        \caption{Ne hagyja a programozót lógni áram alatt.}
    \end{subfigure}\hfill
    \begin{subfigure}{0.45\textwidth}
        \centering
        \includegraphics[width=\textwidth]{image28.png}
        \caption{Kerülje a pogo pinek alatti vezető felületeket.}
    \end{subfigure}
    \caption{Jellemző hibás használati esetek, amelyek másodpercek alatt tönkretehetik az elektronikát.}
\end{figure}
