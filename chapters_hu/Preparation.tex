\chapter{Előkészítés és telepítés}\label{ch:preparation}

Készítse elő a munkaterületet, mielőtt kibontaná az indikátort. A pogo csatlakozónak tisztának és szigeteltnek kell maradnia, a kábelkorbácsnak pedig készen kell állnia a kívánt érzékelőhöz.

\section{Munkaterületi ellenőrzőlista}
\begin{enumerate}
    \item Válassza le a jármű akkumulátorát, és helyezze az indikátort szigetelő alátétre. Tartsa távol a pogo pinektől a fém laptopházakat és más vezető tárgyakat.
    \item Próbálja ki a beszerelési helyet, hogy a kábel elérje a célérzékelőt feszülés nélkül. Vezesse a kábeleket távol éles szélektől vagy nagy hőmérsékletű motortéri elemekről.
    \item Szerelje be a készlethez mellékelt foglalatot, és ellenőrizze, hogy a programozó később a környező burkolat eltávolítása nélkül csatlakoztatható legyen.
\end{enumerate}

\section{Az érzékelő kábelkorbács előkészítése}
\begin{itemize}
    \item Az adott előbeállításhoz megfelelő csatlakozót használja. A barométer VAG \texttt{8K0973703} dugót igényel, míg a turbónyomás-érzékelőhöz tömített, hárompólusú csatlakozó szükséges, például a Bosch \texttt{1928403966}.
    \item Lambda műszer beépítésekor tervezzen két oxigénszondával: a gyári keskenysávú érzékelő az ECU-hoz, valamint egy Bosch LSU~4.9 egy dedikált vezérlőhöz. A vezérlő analóg kimenetét vezesse be az indikátor kábelkorbácsába.
    \item Minden forrasztást és elágazást lásson el tehermentesítéssel és szigeteléssel. A pogo csatlakozó és az érzékelővezetékek nem hajlításra készültek, miközben az indikátor áram alatt van.
\end{itemize}

\section{Támogatás}
A PHOL-LABS Kft egyszeri segítséget nyújt a bekötéssel kapcsolatos kérdésekhez, és díjazás ellenében bővített konzultációt kínál, ha egyedi érzékelő illesztésére van szükség. Ha bármelyik bekötési lépés bizonytalan, még a rendszer áram alá helyezése előtt vegye fel a kapcsolatot a támogatással.
