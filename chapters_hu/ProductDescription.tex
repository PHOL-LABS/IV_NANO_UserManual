\chapter{Termékáttekintés}\label{ch:product-description}

Az IV-Indicators Nano modulok kompakt gombműszerek közös házzal és cserélhető előbeállításokkal. A hardver háromdarabos készletként vagy egyetlen mérési feladatra szánt dedikált indikátorként is megvásárolható.

\section{Elérhető változatok}
\begin{itemize}
    \item \textbf{IV-Indicators műszerszett.} Teljes csomag voltmérő, barométer és hőmérő indikátorral azoknak, akik egységes megjelenésű készletet szeretnének.
    \item \textbf{IV-Indicators Voltmeter.} Előbeállításai a jármű rendszerfeszültségét figyelik vonalas vagy pontszerű kijelzéssel.
    \item \textbf{IV-Indicators Barometer.} A VAG 03C906051A érzékelőre hangolt profilokkal célzott szívó- vagy üzemanyagnyomás-méréshez.
    \item \textbf{IV-Indicators Thermometer.} Gyárilag kalibrálva a Volkswagen klasszikusokban használt Ossca 01176 NTC hőmérőhöz.
    \item \textbf{IV-Indicators Lambda és Boost.} Speciális műszerek lég-üzemanyag arány és turbónyomás figyeléséhez a javasolt érzékelőkkel és vezérlőkkel párosítva.
\end{itemize}

\begin{figure}[htbp]
    \centering
    \begin{subfigure}{0.3\textwidth}
        \centering
        \includegraphics[width=\textwidth]{image33.jpg}
        \caption{Voltmérő indikátor}
    \end{subfigure}\hfill
    \begin{subfigure}{0.3\textwidth}
        \centering
        \includegraphics[width=\textwidth]{image36.jpg}
        \caption{Barométer indikátor}
    \end{subfigure}
    \par\medskip
    \begin{subfigure}{0.3\textwidth}
        \centering
        \includegraphics[width=\textwidth]{image35.jpg}
        \caption{Hőmérő indikátor}
    \end{subfigure}\hfill
    \begin{subfigure}{0.3\textwidth}
        \centering
        \includegraphics[width=\textwidth]{image39.jpg}
        \caption{Lambda és turbónyomás változatok}
    \end{subfigure}
    \caption{A PHOL-LABS Kft által szállított fő IV-Indicators Nano termékváltozatok.}
\end{figure}

\section{Közös képességek}
Minden indikátor átkonfigurálható a mellékelt programozóval és az IV-Conf szoftverrel. A tulajdonosok állíthatják a színátmeneteket, legfeljebb négy szegmensváltást definiálhatnak, valamint választhatnak pont- vagy vonalas megjelenítést. A kalibrációs rutinok lehetővé teszik az utángyártott érzékelőkhöz való illesztést vagy egyedi feszültségtartományok beállítását. A konfiguráció az eszközön tárolódik, így a választott profil áramtalanítás után is megmarad.
