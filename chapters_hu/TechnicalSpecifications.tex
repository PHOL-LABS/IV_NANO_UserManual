\chapter{Műszaki specifikációk}\label{ch:technical-specifications}

\section{Támogatott érzékelők}

\noindent\textbf{Gyári előbeállítások és kompatibilis érzékelők}\label{tab:sensor-characteristics}\\

\begin{samepage}
{\scriptsize
\begin{tblr}{
    colspec={Q[l,2.6cm] Q[l,5cm] Q[l,2.2cm] X[l]},
    hlines
}
\textbf{Indikátor} & \textbf{Érzékelő} & \textbf{Tápellátás} & \textbf{Megjegyzések} \\
Voltmeter & Jármű kábelkorbács & 12~V rendszer & A töltőfeszültség közvetlen mérése. \\
Barometer & VAG \texttt{03C906051A} & 5~V &
    10~bar tartomány, 0{,}4--0{,}5~V névleges kimenet, M10\,\texttimes\,1.0 menet. \\
Thermometer & Ossca \texttt{01176} NTC & Passzív &
    Referenciaértékek az \autoref{app:temperature-table} táblázatban. \\
Lambda & Keskenysávú O$_2$ érzékelő & Passzív &
    0{,}1--0{,}9~V tartomány, 0--5~V-ot fogad szélessávú vezérlőktől. \\
Boost & Dacia \texttt{223657266R} & 5~V &
    20--250~kPa tartomány, 1{,}8~V névleges kimenet, egyetlen rögzítőfurat. \\
\end{tblr}}
\end{samepage}

\section{Érzékelő kompatibilitási táblázatok}
Az alábbi táblázatok összefoglalják a kompatibilis érzékelőket és a gyári előbeállítások kalibrációs referenciaértékeit.

\subsection{Olaj-/hűtőfolyadék-hőmérséklet érzékelők}
\noindent\textbf{Olaj-/hűtőfolyadék-hőmérséklet érzékelők kompatibilitása}\label{tab:oil-coolant-sensors}\\

\begin{samepage}
{\scriptsize
\begin{tblr}{
    colspec={Q[l,6cm] Q[l,3.2cm] X[l]},
    hlines
}
\textbf{Kompatibilis érzékelő} & \textbf{Jel/tápellátás} & \textbf{Kalibrációs paraméterek} \\
Ossca \texttt{01176} NTC termisztor (hőmérő előbeállítás) & Passzív NTC &
    Ellenőrizze az Ossca \texttt{01176} ellenállás-táblázat alapján (pl. 270{,}0~$\Omega$ 58~$^\circ$C-nál, 15{,}9~$\Omega$ 160~$^\circ$C-nál). Névleges kalibráció: $R_{25} = 1$~k$\Omega$, $\beta = 3950$. Egyedi NTC szondáknál frissítse a béta együtthatót és a névleges ellenállást az IV-Conf-ben. \\
RIDEX \texttt{829S0003} (Opel/BMW/Volvo kompatibilis) & 2-pólusú NTC, M12\,\texttimes\,1.5 menet &
    Névleges ellenállás 2080~$\Omega$ 25~$^\circ$C-on és 294~$\Omega$ 80~$^\circ$C-on ($\beta \approx 3740$~K). Működési tartomány 25--80~$^\circ$C. Szenzortest $\varnothing$~7{,}4~mm, 19~mm hatszög, tömítőgyűrűvel. \\
\end{tblr}}
\end{samepage}

\subsection{Levegőhőmérséklet érzékelők}
\noindent\textbf{Levegőhőmérséklet érzékelők kompatibilitása}\label{tab:air-temperature-sensors}\\

\begin{samepage}
{\scriptsize
\begin{tblr}{
    colspec={Q[l,6cm] Q[l,3.2cm] X[l]},
    hlines
}
\textbf{Kompatibilis érzékelő} & \textbf{Jel/tápellátás} & \textbf{Kalibrációs paraméterek} \\
MFA külső levegőhőmérséklet-érzékelő (VAG \texttt{171 919 379 A}, Golf Mk2/Jetta II/Passat B2/B3) & 2-vezetékes NTC, lebegő &
    Kijelzési tartomány kb. $-40^\circ$C és $+96^\circ$C között. Empirikus kalibráció: $R_{25} \approx 510~\Omega$, $\beta \approx 3400$--3500~K (illesztés $+4^\circ$C és $+50^\circ$C között). Diagnosztikai referencia: 200~$\Omega \approx +50^\circ$C. Csatlakozóház \texttt{1J0973802}; párja \texttt{1J0973702}. \\
Ossca \texttt{01176}-tal egyenértékű NTC termisztor (hőmérő előbeállítás) & Passzív NTC &
    Használja az Ossca \texttt{01176} ellenállás-táblázatot referenciaként, és az IV-Conf-ben állítsa be a béta együtthatót/névleges ellenállást az egyedi NTC érzékelőkhöz. \\
\end{tblr}}
\end{samepage}

\subsection{Olajnyomás-érzékelők}
\noindent\textbf{Olajnyomás-érzékelők kompatibilitása}\label{tab:oil-pressure-sensors}\\

\begin{samepage}
{\scriptsize
\begin{tblr}{
    colspec={Q[l,6cm] Q[l,3.2cm] X[l]},
    hlines
}
\textbf{Kompatibilis érzékelő} & \textbf{Jel/tápellátás} & \textbf{Kalibrációs paraméterek} \\
VAG \texttt{03C906051A} nyomásérzékelő (barométer előbeállítás) & 5~V tápellátás, analóg kimenet &
    10~bar abszolút tartomány, 0{,}4--0{,}5~V névleges kimenet nyugalomban. Állítsa be a mérési tartományt a barométer előbeállítással, és szükség esetén módosítsa a min./max. értékeket IV-Conf-ben. \\
\end{tblr}}
\end{samepage}

\subsection{Üzemanyagszint jeladók}
\noindent\textbf{Üzemanyagszint-érzékelők kompatibilitása}\label{tab:fuel-level-sensors}\\

\begin{samepage}
{\scriptsize
\begin{tblr}{
    colspec={Q[l,6cm] Q[l,3.2cm] X[l]},
    hlines
}
\textbf{Kompatibilis érzékelő} & \textbf{Jel/tápellátás} & \textbf{Kalibrációs paraméterek} \\
Egyedi érzékelő (nincs megadott típus) & Az érzékelőtől függ &
    Az IV-Conf-ben állítsa be a minimális és maximális mérési tartományt az installált érzékelőhöz. \\
\end{tblr}}
\end{samepage}

\subsection{Turbónyomás érzékelők}
\noindent\textbf{Turbónyomás érzékelők kompatibilitása}\label{tab:boost-pressure-sensors}\\

\begin{samepage}
{\scriptsize
\begin{tblr}{
    colspec={Q[l,6cm] Q[l,3.2cm] X[l]},
    hlines
}
\textbf{Kompatibilis érzékelő} & \textbf{Jel/tápellátás} & \textbf{Kalibrációs paraméterek} \\
Dacia \texttt{223657266R} (más néven \texttt{161B0004}/\texttt{8200225971}) & 5~V tápellátás, analóg kimenet &
    20--250~kPa működési tartomány, $\sim$1{,}8~V névleges kimenet töltőnyomásnál. Használja a turbónyomás előbeállítást, és szükség esetén igazítsa a Range min/max értékeket. \\
\end{tblr}}
\end{samepage}

\subsection{Szélessávú lambda}
\noindent\textbf{Szélessávú lambda vezérlők kompatibilitása}\label{tab:wideband-lambda}\\

\begin{samepage}
{\scriptsize
\begin{tblr}{
    colspec={Q[l,6cm] Q[l,3.2cm] X[l]},
    hlines
}
\textbf{Kompatibilis érzékelő} & \textbf{Jel/tápellátás} & \textbf{Kalibrációs paraméterek} \\
Szélessávú vezérlő (SLC Free, DIY-EFI TinyWB, Sigma Lambda Controller Free~2) Bosch LSU~4.9 szondával & Lineáris 0--5~V kimenet az indikátornak &
    A szélessávú vezérlők a LSU~4.9 szondát lineáris 0--5~V jellé alakítják, amelyet az indikátor elfogad; a keskenysávú érzékelők 0{,}1--0{,}9~V tartományt adnak. \\
\end{tblr}}
\end{samepage}

\section{Barométer előbeállítás}
A barométer indikátort a Volkswagen-csoport \texttt{03C906051A} nyomásérzékelőjéhez tervezték.
\begin{itemize}
    \item Tápellátás: szabályozott 5~V.
    \item Mérési tartomány: 10~bar abszolút.
    \item Névleges analóg kimenet: 0{,}4--0{,}5~V nyugalomban.
    \item Mechanikus menet: M10\,\texttimes\,1.0.
    \item Csatlakozó: VAG \texttt{8K0973703}.
\end{itemize}

\section{Hőmérő előbeállítás}
A hőmérő előbeállítások gyárilag az Ossca\,01176 NTC termisztorra vannak kalibrálva, amelyet gyakran használnak a Golf~II hűtőrendszerében. Az alternatív szondák ellenőrzéséhez használja az \autoref{app:temperature-table} ellenállás-táblázatot. Egyedi NTC érzékelő esetén az IV-Conf eszközökben frissítse a béta együtthatót és a névleges ellenállást.

\section{Turbónyomás előbeállítás}
A turbónyomás indikátor a Dacia \texttt{223657266R} (\texttt{161B0004}/\texttt{8200225971}) érzékelőt várja.
\begin{itemize}
    \item Hárompólusú csatlakozó (használjon tömített Bosch \texttt{1928403966} vagy azzal egyenértékűt).
    \item Tápfeszültség: +5~V.
    \item Működési tartomány: 20--250~kPa.
    \item Analóg kimenet: körülbelül +1{,}8~V névleges töltőnyomásnál.
\end{itemize}

\section{Lambda integráció}
Az IV-Indicators Lambda közvetlenül olvassa a keskenysávú érzékelőket, de hangolt motoroknál szélessávú vezérlővel együtt használható. Ajánlott vezérlők: SLC Free, DIY-EFI TinyWB és Sigma Lambda Controller Free~2. Ezek a Bosch LSU~4.9 szondát lineáris 0--5~V jellé alakítják, amelyet az indikátor meg tud jeleníteni.

\section{Érzékelő referencia képek}
\begin{figure}[htbp]
    \centering
    \begin{subfigure}{0.3\textwidth}
        \centering
        \includegraphics[width=\textwidth]{image41.jpg}
        \caption{VAG 03C906051A nyomásérzékelő}
    \end{subfigure}\hfill
    \begin{subfigure}{0.3\textwidth}
        \centering
        \includegraphics[width=\textwidth]{image9.png}
        \caption{Barométer kábelkorbács kiosztása}
    \end{subfigure}\hfill
    \begin{subfigure}{0.3\textwidth}
        \centering
        \includegraphics[width=\textwidth]{image42.jpg}
        \caption{Turbónyomás érzékelő bekötési példa}
    \end{subfigure}
    \caption{Referencia hardver a barométer és a turbónyomás előbeállításokhoz.}
\end{figure}
