\chapter{Konfigurálás az IV-Conf Android bétával}\label{ch:android-config}

A készlet részeként elérhető az IV-Conf Android alkalmazás kísérleti változata. A béta alkalmas előbeállítások felírására és színsémák módosítására, de kritikus beállításokhoz a PHOL-LABS a webes csomagot javasolja.

\section{A programozó csatlakoztatása}
\begin{enumerate}
    \item Helyezze be a külső programozót az indikátorba, mielőtt bármilyen kábelt csatlakoztatna.
    \item USB-C OTG kábellel csatlakoztassa a programozót az Android telefonhoz. Ha a készülék más csatlakozót használ, iktasson be USB hubot vagy USB-A átalakítót a telefon és a programozó közé.
    \item Indítsa el az IV-Conf bétát, erősítse meg az USB-engedélykérést, majd érintse meg a \texttt{USB Connect} gombot. Várjon, amíg az állapot \texttt{Connected} lesz.
\end{enumerate}

\section{Előbeállítások és módok betöltése}
\begin{enumerate}
    \item Nyissa meg a menüt (három vonal ikon), és válassza a \texttt{Settings} pontot.
    \item Válasszon a fizikai indikátornak megfelelő előbeállítást — Volts, Baro, Thermo, Lambda vagy Boost.
    \item Érintse meg a \texttt{Set parameters} gombot, és várja meg az átvitel befejezését. Az indikátort stabilan kell tartani az adatküldés alatt.
    \item A képernyőn lévő kapcsolókkal válthat pont- és vonalas kijelzés között, illetve állíthatja a háttérvilágítás csúszkát.
\end{enumerate}

\section{Gradiens szerkesztése}
\begin{enumerate}
    \item Válassza ki a gradiens stílust (alapértelmezés: \texttt{Gradient}).
    \item Adja meg a szegmensek számát (legfeljebb négy), és rendeljen színt és töréspontot mindegyikhez.
    \item Újra nyomja meg a \texttt{Set parameters} gombot, hogy az indikátor megkapja a frissített gradienset.
\end{enumerate}

\section{Felület áttekintése}
\begin{table}[htbp]
    \centering
    \caption{Az IV-Conf Android béta fő csatlakozási és indítóképernyői.}
    \label{tab:android-connection}
    \begin{tabular}{@{} c c c @{}}
        \subcaptionbox{Programozó bekötése}{\includegraphics[width=0.3\textwidth]{image47.png}} &
        \subcaptionbox{Alkalmazás indítóképernyő}{\includegraphics[width=0.3\textwidth]{image17.png}} &
        \subcaptionbox{Sikeres USB kapcsolat}{\includegraphics[width=0.3\textwidth]{image18.png}} \\
    \end{tabular}
\end{table}

\begin{table}[htbp]
    \centering
    \caption{Előbeállítás-, mód- és gradiensvezérlők az IV-Conf Android bétában.}
    \label{tab:android-presets}
    \begin{tabular}{@{} c c c @{}}
        \subcaptionbox{Előbeállítás választás}{\includegraphics[width=0.3\textwidth]{image19.png}} &
        \subcaptionbox{Mód konfiguráció}{\includegraphics[width=0.3\textwidth]{image20.png}} &
        \subcaptionbox{Háttérfény szabályzás}{\includegraphics[width=0.3\textwidth]{image21.png}} \\
        \multicolumn{3}{c}{\subcaptionbox{Gradiens szerkesztő}{\includegraphics[width=0.3\textwidth]{image23.png}}}
    \end{tabular}
\end{table}

\section{Béta korlátai}
A béta alkalmazás fejlesztés alatt áll. Számítson felületváltozásokra és esetenkénti stabilitási problémákra; ha egy beállítás nem érvényesül, térjen vissza a webes csomaghoz. A reprodukálható hibákat közvetlenül a PHOL-LABS-nak jelezze, hogy a firmware és az alkalmazás frissítései összehangolhatók legyenek.
