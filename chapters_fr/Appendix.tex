\chapter{Appendix}\label{ch:appendix}

\section{Reprogrammation IV-Conf web pack}\label{app:reflashing}
Utilisez le pack Web pour les mises à jour du micrologiciel. Réalisez les étapes ci-dessous sans couper l'alimentation une fois le transfert lancé.
\begin{enumerate}
    \item Ouvrez \url{https://phol-labs.com/iv} et connectez-vous au programmateur comme décrit dans \autoref{ch:web-config}.
    \item Assurez-vous que le débit est réglé sur 115200 et cliquez sur \texttt{Connect MODBUS}. Attendez l'affichage du statut en vert.
    \item Appuyez sur \texttt{Stop Send Data} pour interrompre toute transmission automatique de préréglage.
    \item Cliquez sur l'icône du micrologiciel, choisissez le fichier de mise à jour fourni avec votre indicateur et confirmez l'invite.
    \item Laissez l'indicateur immobile jusqu'à ce que la barre de progression se termine et que l'appareil redémarre.
\end{enumerate}

\begin{figure}[htbp]
    \centering
    \begin{subfigure}{0.3\textwidth}
        \centering
        \includegraphics[width=\textwidth]{image15.png}
        \caption{Ouvrir le pack Web}
    \end{subfigure}\hfill
    \begin{subfigure}{0.3\textwidth}
        \centering
        \includegraphics[width=\textwidth]{image11.png}
        \caption{Choisir le port série}
    \end{subfigure}\hfill
    \begin{subfigure}{0.3\textwidth}
        \centering
        \includegraphics[width=\textwidth]{image1.png}
        \caption{Autoriser l'accès}
    \end{subfigure}
    \par\medskip
    \begin{subfigure}{0.3\textwidth}
        \centering
        \includegraphics[width=\textwidth]{image8.png}
        \caption{Régler 115200 bauds}
    \end{subfigure}\hfill
    \begin{subfigure}{0.3\textwidth}
        \centering
        \includegraphics[width=\textwidth]{image25.png}
        \caption{Arrêter les transferts actifs}
    \end{subfigure}\hfill
    \begin{subfigure}{0.3\textwidth}
        \centering
        \includegraphics[width=\textwidth]{image14.png}
        \caption{Connecter MODBUS}
    \end{subfigure}
    \par\medskip
    \begin{subfigure}{0.3\textwidth}
        \centering
        \includegraphics[width=\textwidth]{image6.png}
        \caption{Ouvrir la boîte de dialogue firmware}
    \end{subfigure}\hfill
    \begin{subfigure}{0.3\textwidth}
        \centering
        \includegraphics[width=\textwidth]{image46.png}
        \caption{Choisir le paquet de mise à jour}
    \end{subfigure}\hfill
    \begin{subfigure}{0.3\textwidth}
        \centering
        \includegraphics[width=\textwidth]{image7.png}
        \caption{Confirmer le téléversement}
    \end{subfigure}
    \par\medskip
    \begin{subfigure}{0.3\textwidth}
        \centering
        \includegraphics[width=\textwidth]{image12.png}
        \caption{Attendre la fin}
    \end{subfigure}
    \caption{Suite d'écrans du pack Web IV-Conf utilisés lors de la reprogrammation du micrologiciel.}
\end{figure}

\section{Référence du capteur de température}\label{app:temperature-table}
Le tableau suivant liste les valeurs nominales de résistance pour le capteur NTC Ossca \texttt{01176} utilisé avec le préréglage thermomètre.

\begin{table}[htbp]
    \centering
    \caption{Référence de résistance NTC}
    \label{tab:ntc-reference}
    \begin{tabular}{@{} rr @{}}
        \toprule
        Résistance (\si{\ohm}) & Température (\si{\celsius}) \\
        \midrule
        270.0 & 58 \\
        220.0 & 63 \\
        199.8 & 66 \\
        111.0 & 83 \\
        73.8 & 98 \\
        55.0 & 108 \\
        48.8 & 113 \\
        44.0 & 117 \\
        37.2 & 124 \\
        32.1 & 130 \\
        28.2 & 136 \\
        25.1 & 141 \\
        22.7 & 146 \\
        20.4 & 151 \\
        19.0 & 155 \\
        18.8 & 155 \\
        15.9 & 160 \\
        \bottomrule
    \end{tabular}
\end{table}

Les révisions futures de cette annexe incluront des données supplémentaires sur les capteurs et des schémas de câblage.
