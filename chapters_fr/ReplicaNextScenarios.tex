\chapter{Configuring with the IV-Conf web pack}\label{ch:web-config}

Le pack Web IV-Conf est l'outil principal pour configurer le matériel IV-Indicators Nano. Ouvrez \url{https://phol-labs.com/iv} dans un navigateur basé sur Chromium avec prise en charge Web Serial et connectez le programmateur fourni via USB.

\section{Connexion à l'indicateur}
\begin{enumerate}
    \item Cliquez sur l'icône de port série et sélectionnez l'entrée correspondant au programmateur.
    \item Ouvrez le panneau \texttt{Settings} et réglez le débit sur 115200.
    \item Appuyez sur \texttt{Connect MODBUS}. La bannière d'état doit confirmer la connexion sans indicateur d'erreur rouge.
\end{enumerate}

\section{Catalogue de préréglages}
Plusieurs préréglages existent pour chaque type d'indicateur. Le voltmètre, par exemple, comprend des variantes avec graphiques multi-barres ou des accents jaunes. Sélectionnez un préréglage qui correspond à l'indicateur physique avant de téléverser les réglages.

\begin{table}[htbp]
    \centering
    \caption{Exemples de préréglages voltmètre}
    \label{tab:voltmeter-presets}
    \begin{tabular}{@{} l l @{}}
        \toprule
        Nom du préréglage & Description \\
        \midrule
        IV-Volts-Bar-Multi & Voltmètre avec affichage multi-barres.\\
        IV-Volts-Bar-Multi4 & Variante multi-barres à quatre niveaux.\\
        IV-Volts-BarYellow & Affichage à barres avec un thème jaune.\\
        IV-Volts-Bar & Affichage à barres standard.\\
        IV-Volts & Préréglage minimal sans graphiques supplémentaires.\\
        \bottomrule
    \end{tabular}
\end{table}

\section{Téléversement et téléchargement de données}
\begin{itemize}
    \item Utilisez le bouton \texttt{Send data} pour envoyer le préréglage en cours ou les ajustements manuels à l'indicateur. Le transfert se termine automatiquement en 10--15~secondes.
    \item Cliquez sur \texttt{Get data} pour lire la configuration existante de l'indicateur avant de modifier les plages ou les couleurs.
    \item Le contrôle \texttt{Stop Send Data} annule un téléversement en cours si vous avez sélectionné le mauvais préréglage ou devez modifier des paramètres.
\end{itemize}

Le réglage manuel permet d'adapter la teinte de début/fin du dégradé, l'intensité du rétroéclairage et les plages de capteurs numériques pour du matériel personnalisé. Vérifiez que chaque changement correspond à l'indicateur installé afin d'éviter des mesures trompeuses.

\section{Aperçu de l'interface}
\begin{figure}[htbp]
    \centering
    \begin{subfigure}{0.3\textwidth}
        \centering
        \includegraphics[width=\textwidth]{image15.png}
        \caption{Lancement du pack Web}
    \end{subfigure}\hfill
    \begin{subfigure}{0.3\textwidth}
        \centering
        \includegraphics[width=\textwidth]{image11.png}
        \caption{Sélection du port série}
    \end{subfigure}\hfill
    \begin{subfigure}{0.3\textwidth}
        \centering
        \includegraphics[width=\textwidth]{image1.png}
        \caption{Autorisation d'accès au port}
    \end{subfigure}
    \par\medskip
    \begin{subfigure}{0.3\textwidth}
        \centering
        \includegraphics[width=\textwidth]{image8.png}
        \caption{Réglage du débit}
    \end{subfigure}\hfill
    \begin{subfigure}{0.3\textwidth}
        \centering
        \includegraphics[width=\textwidth]{image22.png}
        \caption{Connexion via MODBUS}
    \end{subfigure}\hfill
    \begin{subfigure}{0.3\textwidth}
        \centering
        \includegraphics[width=\textwidth]{image5.png}
        \caption{Sélecteur de préréglage}
    \end{subfigure}
    \par\medskip
    \begin{subfigure}{0.3\textwidth}
        \centering
        \includegraphics[width=\textwidth]{image3.png}
        \caption{Commandes de réglage manuel}
    \end{subfigure}\hfill
    \begin{subfigure}{0.3\textwidth}
        \centering
        \includegraphics[width=\textwidth]{image2.png}
        \caption{Confirmation d'envoi de données}
    \end{subfigure}
    \caption{Panneaux clés du pack Web IV-Conf utilisés pour la configuration quotidienne.}
\end{figure}

\section{Notes d'utilisation}
Dans la plupart des cas, seuls les panneaux de configuration sont nécessaires. Les mises à jour du micrologiciel sont gérées séparément dans \autoref{ch:appendix}. Si un réglage ne s'applique pas, déconnectez puis reconnectez la session MODBUS et assurez-vous que le port série et le débit corrects sont sélectionnés.
