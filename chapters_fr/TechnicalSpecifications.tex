\chapter{Technical specifications}\label{ch:technical-specifications}

\section{Capteurs pris en charge}

\begin{table}[htbp]
    \centering
    \caption{Préréglages d'usine et capteurs compatibles}
    \label{tab:sensor-characteristics}

    \begin{tabularx}{\textwidth}{@{} l l l X @{}}
        \toprule
        Indicateur & Capteur & Alimentation & Remarques \\
        \midrule
        Voltmètre & Faisceau du véhicule & Système 12~V & Mesure directe de la tension de charge. \\
        Baromètre & VAG \texttt{03C906051A} & 5~V &
            Plage 10~bar, sortie nominale 0.4--0.5~V, filetage M10\,\texttimes\,1.0. \\
        Thermomètre & NTC Ossca \texttt{01176} & Passif &
            Valeurs de référence listées dans \autoref{app:temperature-table}. \\
        Lambda & Sonde O$_2$ étroite & Passif &
            Variation 0.1--0.9~V, accepte 0--5~V depuis les contrôleurs large bande. \\
        Boost & Dacia \texttt{223657266R} & 5~V &
            Étendue 20--250~kPa, sortie nominale 1.8~V, seul perçage de montage. \\
        \bottomrule
\end{tabularx}
\end{table}

\section{Tableaux de compatibilité des capteurs}
Les tableaux suivants résument les capteurs compatibles et les références d'étalonnage des préréglages d'usine.

\subsection{Capteurs de température d'huile/liquide de refroidissement}
\begin{table}[htbp]
    \centering
    \caption{Compatibilité des capteurs de température d'huile/liquide de refroidissement}
    \label{tab:oil-coolant-sensors}
    \begin{tabularx}{\textwidth}{@{} l l X @{}}
        \toprule
        Capteur compatible & Signal/alimentation & Paramètres d'étalonnage \\
        \midrule
        Thermistance NTC Ossca \texttt{01176} (préréglages thermomètre) & NTC passif &
            Valider avec la table de résistances Ossca \texttt{01176} (p.\,ex., 270{,}0~$\Omega$ à 58~$^\circ$C jusqu'à 15{,}9~$\Omega$ à 160~$^\circ$C). Étalonnage nominal : $R_{25} = 1$~k$\Omega$, $\beta = 3950$. Pour des sondes NTC personnalisées, ajustez le beta et la résistance nominale dans IV-Conf. \\
        RIDEX \texttt{829S0003} (compatible Opel/BMW/Volvo) & NTC à 2 broches, filetage M12\,\texttimes\,1.5 &
            Résistance nominale 2080~$\Omega$ à 25~$^\circ$C et 294~$\Omega$ à 80~$^\circ$C ($\beta \approx 3740$~K). Plage de fonctionnement 25--80~$^\circ$C. Corps $\varnothing$~7{,}4~mm, hexagone 19~mm, joint d'étanchéité fourni. \\
        \bottomrule
    \end{tabularx}
\end{table}

\subsection{Capteurs de température d'air}
\begin{table}[htbp]
    \centering
    \caption{Compatibilité des capteurs de température d'air}
    \label{tab:air-temperature-sensors}
    \begin{tabularx}{\textwidth}{@{} l l X @{}}
        \toprule
        Capteur compatible & Signal/alimentation & Paramètres d'étalonnage \\
        \midrule
        Sonde de température extérieure MFA (VAG \texttt{171 919 379 A}, Golf Mk2/Jetta II/Passat B2/B3) & NTC à 2 fils, flottant &
            Plage d'affichage approximative de $-40^\circ$C à $+96^\circ$C. Étalonnage empirique : $R_{25} \approx 510~\Omega$, $\beta \approx 3400$--3500~K (ajusté entre $+4^\circ$C et $+50^\circ$C). Référence de diagnostic : 200~$\Omega \approx +50^\circ$C. \\
        Thermistance NTC équivalente à Ossca \texttt{01176} (préréglages thermomètre) & NTC passif &
            Utilisez la table de résistances Ossca \texttt{01176} comme référence et ajustez le beta/la résistance nominale pour les capteurs NTC personnalisés via IV-Conf. \\
        \bottomrule
    \end{tabularx}
\end{table}

\subsection{Capteurs de pression d'huile}
\begin{table}[htbp]
    \centering
    \caption{Compatibilité des capteurs de pression d'huile}
    \label{tab:oil-pressure-sensors}
    \begin{tabularx}{\textwidth}{@{} l l X @{}}
        \toprule
        Capteur compatible & Signal/alimentation & Paramètres d'étalonnage \\
        \midrule
        Capteur de pression VAG \texttt{03C906051A} (préréglages baromètre) & Alimentation 5~V, sortie analogique &
            Plage absolue 10~bar, sortie nominale 0.4--0.5~V au repos. Configurez la plage avec le préréglage baromètre et ajustez les min./max. dans IV-Conf si nécessaire. \\
        \bottomrule
    \end{tabularx}
\end{table}

\subsection{Jauges de niveau de carburant}
\begin{table}[htbp]
    \centering
    \caption{Compatibilité des capteurs de niveau de carburant}
    \label{tab:fuel-level-sensors}
    \begin{tabularx}{\textwidth}{@{} l l X @{}}
        \toprule
        Capteur compatible & Signal/alimentation & Paramètres d'étalonnage \\
        \midrule
        Capteur personnalisé (aucun modèle spécifique) & Dépend du capteur &
            Utilisez IV-Conf pour définir les plages de mesure min./max. en fonction du capteur installé. \\
        \bottomrule
    \end{tabularx}
\end{table}

\subsection{Capteurs de pression de suralimentation}
\begin{table}[htbp]
    \centering
    \caption{Compatibilité des capteurs de pression de suralimentation}
    \label{tab:boost-pressure-sensors}
    \begin{tabularx}{\textwidth}{@{} l l X @{}}
        \toprule
        Capteur compatible & Signal/alimentation & Paramètres d'étalonnage \\
        \midrule
        Dacia \texttt{223657266R} (aussi \texttt{161B0004}/\texttt{8200225971}) & Alimentation 5~V, sortie analogique &
            Fenêtre de fonctionnement 20--250~kPa, sortie nominale $\sim$1{,}8~V en suralimentation. Utilisez le préréglage boost et ajustez range min/max si nécessaire. \\
        \bottomrule
    \end{tabularx}
\end{table}

\subsection{Lambda large bande}
\begin{table}[htbp]
    \centering
    \caption{Compatibilité des contrôleurs lambda large bande}
    \label{tab:wideband-lambda}
    \begin{tabularx}{\textwidth}{@{} l l X @{}}
        \toprule
        Capteur compatible & Signal/alimentation & Paramètres d'étalonnage \\
        \midrule
        Contrôleur large bande (SLC Free, DIY-EFI TinyWB, Sigma Lambda Controller Free~2) avec sonde Bosch LSU~4.9 & Sortie linéaire 0--5~V vers l'indicateur &
            Les contrôleurs large bande traduisent la sonde LSU~4.9 en un signal linéaire 0--5~V accepté par l'indicateur ; les capteurs à bande étroite fournissent une variation de 0.1--0.9~V. \\
        \bottomrule
    \end{tabularx}
\end{table}


\section{Préréglage baromètre}
L'indicateur baromètre est conçu pour le capteur de pression du groupe Volkswagen \texttt{03C906051A}.
\begin{itemize}
    \item Alimentation : 5~V régulés.
    \item Plage de mesure : 10~bar absolus.
    \item Sortie analogique nominale : 0.4--0.5~V au repos.
    \item Filetage mécanique : M10\,\texttimes\,1.0.
    \item Connecteur : VAG \texttt{8K0973703}.
\end{itemize}

\section{Préréglage thermomètre}
Les préréglages thermomètre sont livrés étalonnés pour le thermistor NTC Ossca\,01176 couramment monté sur les systèmes de refroidissement de Golf~II. Utilisez le tableau de résistances de \autoref{app:temperature-table} pour valider des sondes alternatives. Pour des capteurs NTC personnalisés, mettez à jour le coefficient beta et la résistance nominale via les outils IV-Conf.

\section{Préréglage boost}
L'indicateur boost attend le capteur Dacia \texttt{223657266R} (\texttt{161B0004}/\texttt{8200225971}).
\begin{itemize}
    \item Connecteur à trois broches (utilisez un modèle étanche Bosch \texttt{1928403966} ou équivalent).
    \item Tension d'alimentation : +5~V.
    \item Fenêtre de fonctionnement : 20--250~kPa.
    \item Sortie analogique : environ +1.8~V à la suralimentation nominale.
\end{itemize}

\section{Intégration lambda}
IV-Indicators Lambda lit directement les sondes étroites mais est conçu pour fonctionner avec un contrôleur large bande sur les moteurs préparés. Les contrôleurs recommandés incluent SLC Free, DIY-EFI TinyWB et Sigma Lambda Controller Free~2. Chaque appareil convertit une sonde Bosch LSU~4.9 en un signal linéaire 0--5~V que l'indicateur peut afficher.

\section{Images de référence des capteurs}
\begin{figure}[htbp]
    \centering
    \begin{subfigure}{0.3\textwidth}
        \centering
        \includegraphics[width=\textwidth]{image41.jpg}
        \caption{Capteur de pression VAG 03C906051A}
    \end{subfigure}\hfill
    \begin{subfigure}{0.3\textwidth}
        \centering
        \includegraphics[width=\textwidth]{image9.png}
        \caption{Affectation des broches pour le faisceau baromètre}
    \end{subfigure}\hfill
    \begin{subfigure}{0.3\textwidth}
        \centering
        \includegraphics[width=\textwidth]{image42.jpg}
        \caption{Exemple de connexion capteur de boost}
    \end{subfigure}
    \caption{Matériel de référence fourni ou recommandé pour les préréglages baromètre et boost.}
\end{figure}
