\chapter{Preparation and installation}\label{ch:preparation}

Préparez l'espace de travail avant de déballer l'indicateur. Le connecteur pogo doit rester propre et isolé, et le faisceau doit être prêt pour le capteur requis.

\section{Liste de contrôle pour l'atelier}
\begin{enumerate}
    \item Déconnectez la batterie du véhicule et placez l'indicateur sur un tapis non conducteur. Gardez les coques métalliques d'ordinateur portable et autres objets conducteurs éloignés des broches pogo.
    \item Essayez l'emplacement de montage afin que le faisceau puisse atteindre le capteur cible sans contrainte. Acheminer les câbles loin des arêtes vives ou des parties moteur à haute température.
    \item Installez la douille fournie avec le kit et vérifiez que le programmateur pourra être connecté plus tard sans retirer les garnitures environnantes.
\end{enumerate}

\section{Préparer le faisceau du capteur}
\begin{itemize}
    \item Utilisez le connecteur approprié pour le préréglage choisi. Le baromètre nécessite une fiche VAG \texttt{8K0973703}, tandis que le capteur de suralimentation a besoin d'un connecteur étanche à trois broches comme le Bosch \texttt{1928403966}.
    \item Lors de l'installation d'un indicateur lambda, prévoyez deux sondes à oxygène : l'unité étroite d'origine pour l'ECU et une Bosch LSU~4.9 reliée à un contrôleur dédié. Faites entrer la sortie analogique du contrôleur dans le faisceau de l'indicateur.
    \item Assurez un soulagement de traction et une isolation pour chaque épissure. Le connecteur pogo et les fils des capteurs ne sont pas conçus pour fléchir lorsque l'indicateur est alimenté.
\end{itemize}

\section{Assistance}
PHOL-LABS Kft offre une séance d'assistance unique pour les questions de câblage et propose des consultations prolongées payantes lorsque des adaptations de capteurs personnalisées sont nécessaires. Contactez le support avant de mettre le système sous tension si une étape de câblage n'est pas claire.
