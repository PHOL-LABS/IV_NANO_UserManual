\chapter{Configuring with the IV-Conf Android beta}\label{ch:android-config}

Une application Android IV-Conf expérimentale est fournie avec le jeu d'indicateurs. La bêta est fonctionnelle pour flasher des préréglages et modifier les jeux de couleurs, mais PHOL-LABS recommande le pack Web pour les réglages critiques.

\section{Connexion du programmateur}
\begin{enumerate}
    \item Insérez le programmateur externe dans l'indicateur avant de connecter tout câble.
    \item Reliez le programmateur au téléphone Android avec un câble USB-C OTG. Si le téléphone utilise un connecteur différent, placez un hub USB ou un adaptateur USB-A entre le téléphone et le programmateur.
    \item Lancez la bêta IV-Conf, confirmez la demande d'autorisation USB et appuyez sur \texttt{USB Connect}. Attendez que le statut affiche \texttt{Connected}.
\end{enumerate}

\section{Chargement des préréglages et des modes}
\begin{enumerate}
    \item Ouvrez le menu latéral (icône trois barres) et sélectionnez \texttt{Settings}.
    \item Choisissez un préréglage correspondant à l'indicateur physique — Volts, Baro, Thermo, Lambda ou Boost.
    \item Touchez \texttt{Set parameters} et laissez la transmission se terminer. L'indicateur doit rester immobile pendant l'envoi des données.
    \item Utilisez les commutateurs à l'écran pour passer entre les modes d'affichage point ou barre et ajuster le curseur d'intensité du rétroéclairage.
\end{enumerate}

\section{Édition des dégradés}
\begin{enumerate}
    \item Sélectionnez le style de dégradé (par défaut : \texttt{Gradient}).
    \item Choisissez le nombre de segments (jusqu'à quatre) et attribuez des couleurs et positions de transition à chaque segment.
    \item Appuyez à nouveau sur \texttt{Set parameters} pour écrire le dégradé mis à jour dans l'indicateur.
\end{enumerate}

\section{Aperçu de l'interface}
\begin{table}[htbp]
    \centering
    \caption{Écrans clés de connexion et de démarrage de la bêta IV-Conf Android.}
    \label{tab:android-connection}
    \begin{tabular}{@{} c c c @{}}
        \subcaptionbox{Câblage du programmateur}{\includegraphics[width=0.3\textwidth]{image47.png}} &
        \subcaptionbox{Écran de démarrage de l'application}{\includegraphics[width=0.3\textwidth]{image17.png}} &
        \subcaptionbox{Connexion USB réussie}{\includegraphics[width=0.3\textwidth]{image18.png}} \\
    \end{tabular}
\end{table}

\begin{table}[htbp]
    \centering
    \caption{Commandes de préréglage, de mode et de dégradé dans la bêta IV-Conf Android.}
    \label{tab:android-presets}
    \begin{tabular}{@{} c c c @{}}
        \subcaptionbox{Sélection du préréglage}{\includegraphics[width=0.3\textwidth]{image19.png}} &
        \subcaptionbox{Configuration du mode}{\includegraphics[width=0.3\textwidth]{image20.png}} &
        \subcaptionbox{Contrôle du rétroéclairage}{\includegraphics[width=0.3\textwidth]{image21.png}} \\
        \multicolumn{3}{c}{\subcaptionbox{Éditeur de dégradé}{\includegraphics[width=0.3\textwidth]{image23.png}}}
    \end{tabular}
\end{table}

\section{Limitations de la bêta}
L'application bêta est en cours de développement actif. Attendez-vous à des changements d'interface et à des problèmes de stabilité occasionnels ; revenez au pack Web si un réglage n'est pas appliqué correctement. Signalez tout bug reproductible directement à PHOL-LABS afin que les mises à jour du micrologiciel et de l'application puissent être alignées.
