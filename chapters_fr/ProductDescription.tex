\chapter{Product overview}\label{ch:product-description}

Les modules IV-Indicators Nano sont des afficheurs compacts au format bouton avec un boîtier commun et des préréglages interchangeables. Le matériel est vendu en pack de trois ou en indicateurs dédiés à un seul rôle de mesure.

\section{Variantes disponibles}
\begin{itemize}
    \item \textbf{Pack de trois indicateurs IV-Indicators.} Un ensemble complet contenant les indicateurs voltmètre, baromètre et thermomètre pour les conducteurs qui veulent une série assortie.
    \item \textbf{IV-Indicators Voltmètre.} Livré avec des préréglages qui surveillent la tension du système véhicule en affichage à barres ou à points.
    \item \textbf{IV-Indicators Baromètre.} Destiné à la surveillance de la pression de collecteur ou de carburant avec des profils réglés pour le capteur VAG 03C906051A.
    \item \textbf{IV-Indicators Thermomètre.} Calibré en usine pour le capteur de température NTC Ossca 01176 utilisé dans les modèles Volkswagen classiques.
    \item \textbf{IV-Indicators Lambda et Boost.} Indicateurs spécialisés pour le suivi du rapport air-carburant et de la pression de suralimentation lorsqu'ils sont associés aux capteurs et contrôleurs recommandés.
\end{itemize}

\begin{figure}[htbp]
    \centering
    \begin{subfigure}{0.3\textwidth}
        \centering
        \includegraphics[width=\textwidth]{image33.jpg}
        \caption{Indicateur voltmètre}
    \end{subfigure}\hfill
    \begin{subfigure}{0.3\textwidth}
        \centering
        \includegraphics[width=\textwidth]{image36.jpg}
        \caption{Indicateur baromètre}
    \end{subfigure}
    \par\medskip
    \begin{subfigure}{0.3\textwidth}
        \centering
        \includegraphics[width=\textwidth]{image35.jpg}
        \caption{Indicateur thermomètre}
    \end{subfigure}\hfill
    \begin{subfigure}{0.3\textwidth}
        \centering
        \includegraphics[width=\textwidth]{image39.jpg}
        \caption{Versions lambda et boost}
    \end{subfigure}
    \caption{Principales variantes IV-Indicators Nano fournies par PHOL-LABS Kft.}
\end{figure}

\section{Fonctionnalités partagées}
Chaque indicateur peut être reconfiguré à l'aide du programmateur fourni et du logiciel IV-Conf. Les utilisateurs peuvent ajuster les dégradés de couleur, définir jusqu'à quatre transitions de segments et choisir un rendu par points ou par barres. Les routines d'étalonnage permettent d'adapter l'indicateur à des capteurs du marché secondaire ou d'affiner des plages de tension personnalisées. Une fois configurés, les réglages sont stockés sur l'appareil pour qu'il mémorise le profil choisi même après coupure d'alimentation.
